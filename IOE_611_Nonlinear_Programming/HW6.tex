\documentclass[11pt]{article}
\usepackage{geometry}                
\geometry{letterpaper,tmargin=1in,bmargin=1in,lmargin=1in,rmargin=1in} 
\usepackage[parfill]{parskip}    % Activate to begin paragraphs with an empty line rather than an indent
\usepackage{graphicx}
\usepackage{amsmath, amsfonts, amsthm, amssymb} 
\usepackage[shortlabels]{enumitem}
\usepackage{xcolor}
\usepackage{mathtools}

\parskip = 0.1in

\pagestyle{myheadings}
\markright{Homework of IOE 611 Nonlinear Programming\hfill Yulun Zhuang \hfill}

 % some traditional definitions that can be blamed on craig barratt
 \newcommand{\BEAS}{\begin{eqnarray*}}
\newcommand{\EEAS}{\end{eqnarray*}}
\newcommand{\BEA}{\begin{eqnarray}}
\newcommand{\EEA}{\end{eqnarray}}
\newcommand{\BEQ}{\begin{equation}}
\newcommand{\EEQ}{\end{equation}}
\newcommand{\BIT}{\begin{itemize}}
\newcommand{\EIT}{\end{itemize}}

% text abbrevs
\newcommand{\eg}{e.g.}
\newcommand{\ie}{i.e.}

% std math stuff
\newcommand{\ones}{\mathbf 1}
\newcommand{\real}{{{\mathbb{R}}}}
\newcommand{\integer}{{{\mathbb{Z}}}}
\newcommand{\complex}{{{\mathbb{C}}}}
\newcommand{\symm}{{{\mathbb{S}}}}  % symmetric matrices
%
% lin alg stuff
\newcommand{\Span}{\mbox{\textrm{span}}}
\newcommand{\range}{{\mathcal R}}
\newcommand{\nullspace}{{\mathcal N}}
\newcommand{\Rank}{\mathop{\textbf{rank}}}
\newcommand{\Tr}{\mathop{\textbf{tr}}}
\newcommand{\cond}{\mathop{\textbf{cond}}}
\newcommand{\diag}{\mathop{\textbf{diag}}}
\newcommand{\lambdamax}{\lambda_{\rm max}}
\newcommand{\lambdamin}{\lambda_{\rm min}}

% probability stuff
\newcommand{\Prob}{\mathop{\textbf{prob}}}
\newcommand{\Expect}{\mathop{\textbf{E{}}}}
\newcommand{\var}{\mathop{\textbf{var}}} % variance
% not sure why we have \Expect and \Prob but \var ???

% convexity & optimization stuff
\newcommand{\Co}{\mathop {\textbf{conv}}} % convex hull
\newcommand{\argmin}{\mathop{\rm argmin}}
\newcommand{\argmax}{\mathop{\rm argmax}}
\newcommand{\epi}{\mathop{\textbf{epi}}}
%\newcommand{\hypo}{\mathop{\textbf{hypo}}}}

% sup and inf that look OK in saddle-point form!
%\newcommand{\ourinf}{\mathop{\raisebox{0ex}[0ex][.4ex]{\,inf\,}}}
%\newcommand{\oursup}{\mathop{\raisebox{0ex}[0ex][.4ex]{\,sup\,}}}
\newcommand{\ourinf}{\mathop{\,\mathrm{inf}\, {\rule[-.5ex]{0ex}{0ex}}}}
\newcommand{\oursup}{\mathop{\,\mathrm{sup}\, {\rule[-.5ex]{0ex}{0ex}}}}
%makes latex believe that inf and sup both extend .4ex below
%the baseline

\newcommand{\dist}{\mathop{\textbf{dist}}}
\newcommand{\vol}{\mathop{\textbf{vol}}} % volume
\newcommand{\Card}{\mathop{\textbf{card}}} % cardinality
\newcommand{\sign}{\mathop{\textbf{sign}}}

\newcommand{\dom}{\mathop{\textbf{dom}}} % domain
\newcommand{\aff}{\mathop{\textbf{aff}}} % affine hull
\newcommand{\cl}{\mathop{\textbf{cl}}} % closure
\newcommand{\intr}{\mathop{\textbf{int}}} % interior
\newcommand{\relint}{\mathop{\textbf{rel int}}} % relative interior
\newcommand{\bd}{\mathop{\textbf{bd}}} % boundary

%why do we have the following but not \nust?
\newcommand{\xst}{x^\star}
\newcommand{\lambdast}{\lambda^\star}

% defs for cones & generalized inequalities
% these seem kind of awkward; should fix some day
% rewrite them to use args?
\newcommand{\geqK}{\mathrel{\succeq_K}}
\newcommand{\gK}{\mathrel{\succ_K}}
\newcommand{\leqK}{\mathrel{\preceq_K}}
\newcommand{\lK}{\mathrel{\prec_K}}
\newcommand{\geqKst}{\mathrel{\succeq_{K^*}}}
\newcommand{\gKst}{\mathrel{\succ_{K^*}}}
\newcommand{\leqKst}{\mathrel{\preceq_{K^*}}}
\newcommand{\lKst}{\mathrel{\prec_{K^*}}}
\newcommand{\geqL}{\mathrel{\succeq_L}}
\newcommand{\gL}{\mathrel{\succ_L}}
\newcommand{\leqL}{\mathrel{\preceq_L}}
\newcommand{\lL}{\mathrel{\prec_L}}
\newcommand{\geqLst}{\mathrel{\succeq_{L^*}}}
\newcommand{\gLst}{\mathrel{\succ_{L^*}}}
\newcommand{\leqLst}{\mathrel{\preceq_{L^*}}}
\newcommand{\lLst}{\mathrel{\prec_{L^*}}}

%\newcounter{lecture}
%\newcommand{\lecturefl}[1]{   % use with foiltex landscape
%% \addtocounter{lecture}{1}
% \refstepcounter{lecture}
% \setcounter{equation}{0}
% \setcounter{page}{1}
% \renewcommand{\theequation}{\arabic{equation}}
% \renewcommand{\thepage}{\arabic{lecture}--\arabic{page}}
% \raggedright
% \parindent 0pt
% \rightfooter{\thepage}
% \leftheader{}
% \rightheader{}
% \LogoOff
% \input header 
% \begin{center}
%% {\Large \bfseries Lecture \arabic{lecture} \\*[\bigskipamount] {#1}}
%{\Large \bfseries \arabic{lecture}.  {#1}}
% \end{center}
% \MyLogo{#1}
%}

%\newcommand{\lectureflstar}[1]{   % use with foiltex landscape
% \setcounter{equation}{0}
% \setcounter{page}{1}
% \renewcommand{\theequation}{\arabic{equation}}
% \renewcommand{\thepage}{\arabic{page}}
% \raggedright
% \parindent 0pt
% \rightfooter{\thepage}
% \leftheader{}
% \rightheader{}
% \LogoOff
% \input header 
% \begin{center}
% {\Large \bfseries #1}
% \end{center}
% \MyLogo{#1}
%}
%\newcounter{oursection}
%\newcommand{\frametitle}[1]{  % for use with foiltex landscape
% \addtocounter{oursection}{1}
%% \setcounter{equation}{0}
% \foilhead[-1.0cm]{#1}
% \LogoOn
%}

\newenvironment{algdesc}%
   {\begin{list}{}{%
   \setlength{\rightmargin}{0\linewidth}%
   \setlength{\leftmargin}{.05\linewidth}}%
   \sffamily\small
   \item[]{\setlength{\parskip}{0ex}\hrulefill\par%
   \nopagebreak{}}}%
   {{\setlength{\parskip}{-1ex}\nopagebreak\par\hrulefill} \end{list}}

\newenvironment{colm}{\left[\begin{array}{c}}{\end{array}\right]}
\newenvironment{colv}{\left(\begin{array}{c}}{\end{array}\right)}


\newcommand{\oh}{\frac12}
\newcommand{\st}{\text{subject to}}
\newcommand{\gfb}{\nabla f(\bar x)}
\newcommand{\hfb}{H(\bar x)}

\newtheorem{theorem}{Theorem}[section]
\newtheorem{remark}[theorem]{Remark}%[section]
\newtheorem{definition}[theorem]{Definition}%[section]
\newtheorem{proposition}[theorem]{Proposition}%[section]
\newtheorem{lemma}[theorem]{Lemma}%[section]
\newtheorem{corollary}[theorem]{Corollary}%[section]
\newtheorem{assumption}{Assumption}
\newtheorem{claim}{Claim}
\newtheorem{exam}{Example}
\newenvironment{solution}
  {\renewcommand\qedsymbol{$\square$}\begin{proof}[\textbf{Solution}]}
  {\end{proof}}
\renewcommand{\proofname}{\textbf{Proof}}

% Colors
\newcommand{\red}[1]{\textcolor{red}{#1}}
\newcommand{\blue}[1]{\textcolor{blue}{#1}}
\newcommand{\green}[1]{\textcolor{green}{#1}}

% Handy math notations
\newcommand{\grad}{\nabla}
\newcommand{\hess}{\nabla^2}
\newcommand{\tr}{\text{tr}}
\newcommand{\pd}[2][]{ \frac{\partial #1}{\partial #2}} % Partial derivatives
\renewcommand{\d}{{\rm d}}
\newcommand{\ddt}{\frac{\d}{\d t}}
\newcommand{\half}{\frac{1}{2}} % 1/2
\newcommand{\inv}{^{-1}}        % inverse
\newcommand{\T}{^\top}          % transpose

\begin{document}
\title{IOE 611: Homework 6}
\author{Yulun Zhuang}
\maketitle
%**********************************
\section*{Problem 1}
% Exercise 9.12
\textit{Trust region Newton method}.
If $\hess f(x)$ is singular (or very ill-conditioned), the Newton step $\Delta x_{\text{nt}} = -\hess f(x)\inv \grad f(x)$ is not well defined. Instead we can define a search direction $\Delta x_{\text{tr}}$ as the solution of
\begin{align*}
    \text{minimize}\quad & \half v\T H v + g\T v\\
    \text{subject to}\quad & \|v\|_2 \leq \gamma,
\end{align*}
where $H = \hess f(x), g = \grad f(x)$, and $\gamma$ is a positive constant. The point $x + \Delta x_{\text{tr}}$ minimizes the second-order approximation of $f$ at $x$, subject to the constraint that $\| (x + \Delta x_{\text{tr}}) - x\|_2 \leq \gamma$. The set $\{v \mid \|v\|_2\leq \gamma\}$ is called the \textit{trust region}. The parameter $\gamma$, the size of the trust region, reflects our confidence in the second-order model.

Show that $\Delta x_{\text{tr}}$ minimizes
\[
    \half v\T H v + g\T v + \hat \beta \|v\|_2^2,
\]
for some $\hat \beta$. THis quadratic function can be interpreted as a regularized quadratic model for $f$ around $x$.



\clearpage
\section*{Problem 2}
% Exercise 10.15
\textit{Equality constrained entropy maximization}. Consider the equality constrained entropy maximization problem
\begin{align*}
    \text{minimize}\quad & f(x) = \sum_{n}^{i=1} x_i \log x_i\\
    \text{subject to}\quad & Ax = b,
\end{align*}
with $\dom f = \real^n_{++}$ and $A\in\real^{p\times n}$, with $p < n$.

Generate a problem instance with $n = 100$ and $p = 30$ by choosing $A$ randomly (checking that it has full rank), choosing $\hat x$ as a random positive vector (e.g., with entries uniformly distributed on [0, 1]) and then setting $b = A\hat x$. (Thus, $\hat x$ is feasible.)

Compute the solution of the problem using the following methods.

(a) \textit{Standard Newton method}. You can use initial point $x^{(0)} = \hat x$.

(b) \textit{Infeasible start Newton method}. You can use initial point $x^{(0)} = \hat x$ (to compare with the standard Newton method), and also the initial point $x^{(0)} = \ones$.

(c) \textit{Dual Newton method}, i.e. the standard Newton method applied to the dual problem.

Verify that the three methods compute the same optimal point (and Lagrange multiplier).
Compare the computational effort per step for the three methods, assuming relevant
structure is exploited. 
(Your implementation, however, does not need to exploit structure to compute the Newton step.)

\clearpage
\section*{Problem 3}
% Exercise 11.7
\textit{Tangent to central path}. This problem concerns $dx^*(t)/dt$, which gives the tangent to the central path at the point $x^*(t)$. For simplicity, we consider a problem without equality constraints; the results readily generalize to problems with equality constraints.

(a) Find an explicit expression for $dx^*(t)/dt$.

(b) Show that $f_0(x^*(t))$ decreases as $t$ increases. Thus the objective value in the barrier method decreases, as the parameter $t$ is increased. (We already know that the duality gap, which is $m/t$ decreases as t increases.)


\end{document}
