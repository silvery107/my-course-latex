\documentclass[11pt]{article}
\usepackage{geometry}                
\geometry{letterpaper,tmargin=1in,bmargin=1in,lmargin=1in,rmargin=1in} 
\usepackage[parfill]{parskip}    % Activate to begin paragraphs with an empty line rather than an indent
\usepackage{graphicx}
\usepackage{amsmath, amsfonts, amsthm, amssymb} 
\usepackage[shortlabels]{enumitem}
\usepackage{xcolor}
\usepackage{mathtools}

\parskip = 0.1in

\pagestyle{myheadings}
\markright{Homework of IOE 611 Nonlinear Programming\hfill Yulun Zhuang \hfill}

\input{../611defs}

\newcommand{\oh}{\frac12}
\newcommand{\st}{\text{subject to}}
\newcommand{\gfb}{\nabla f(\bar x)}
\newcommand{\hfb}{H(\bar x)}

\newtheorem{theorem}{Theorem}[section]
\newtheorem{remark}[theorem]{Remark}%[section]
\newtheorem{definition}[theorem]{Definition}%[section]
\newtheorem{proposition}[theorem]{Proposition}%[section]
\newtheorem{lemma}[theorem]{Lemma}%[section]
\newtheorem{corollary}[theorem]{Corollary}%[section]
\newtheorem{assumption}{Assumption}
\newtheorem{claim}{Claim}
\newtheorem{exam}{Example}
\newenvironment{solution}
  {\renewcommand\qedsymbol{$\square$}\begin{proof}[\textbf{Solution}]}
  {\end{proof}}
\renewcommand{\proofname}{\textbf{Proof}}

\newcommand{\red}[1]{\textcolor{red}{#1}}
\newcommand{\blue}[1]{\textcolor{blue}{#1}}
\newcommand{\green}[1]{\textcolor{green}{#1}}
\newcommand{\grad}{\nabla}
\newcommand{\hess}{\nabla^2}
\newcommand{\tr}{\text{tr}}

\newcommand{\dd}{\mathrm{d}}
\newcommand{\RR}{\mathbb{R}}
\newcommand{\NN}{\mathbb{N}}
\newcommand{\ZZ}{\mathbb{Z}}
\newcommand{\bS}{\mathbb{S}}

\newcommand{\pd}[2][]{ \frac{\partial #1}{\partial #2}} % Partial derivatives
\renewcommand{\d}{{\rm d}}
\newcommand{\ddt}{\frac{\d}{\d t}}
\newcommand{\half}{\frac{1}{2}}
\newcommand{\inv}{^{-1}}
\newcommand{\T}{^\top}

\begin{document}
\title{IOE 611: Midterm}
\author{Yulun Zhuang}
\maketitle
%**********************************
\section*{Problem 1}
Obtain the dual cone of the set $K \subset \real^3$ defined as:
\[
K = \{(x, y, z):y>0, ye^{x/y} \leq z\}
\]

\begin{solution}
By definition, the dual cone $K^*$ can be written as:
$$
\begin{aligned}
{K}^* & =\{(u, v, w) \mid  x u+y v+z w \geq 0, \forall(x, y, z) \in K\} \\
& =\left\{(u, v, w) \mid  u \frac{x}{y}+v+w \frac{z}{y}>0, \forall y>0, e^{x / y} \leq z / y\right\}
\end{aligned}
$$

Note that if $w<0$, then for any $u, v \in \mathbb{R}$, consider $\left(x_0, y_0,  z_0\right)=(0, 1, \max \{1,-\frac{|v|+1}{w}\}) \in K$, we have
\[
u x_0+v y_0+w z_0=v+w z_0 \leq v+w \cdot\left(-\frac{|v|+1}{w}\right)=v-|v|-1<0
\]
which implies that $(u, v, w)\notin K^*$.

Thus, $\forall(u, v, w) \in {K}^*$, we must have $w \geq 0$.

If $w=0$, consider $y=1, z=e^x$, then $\forall x \in \mathbb{R},(x, y, z) \in K$, therefore if $(u, v, 0) \in K^*$, we must have $u x+v \geq 0$ for all $x \in \mathbb{R}$. 

Then, we must have $u=0, v \geq 0$.

Conversely, if $u=w=0, v\geq 0$, then $\forall(x, y, z) \in {K},  u x+v y+w z=v y \geq 0$.

If $w > 0$, then for any $(x, y, z) \in K$,
$$
u \frac{x}{y}+v+w \frac{z}{y} \geq u \frac{x}{y}+v+w e^{x / y}
$$
and equality holds if and only if $z / y=e^{x / y}$.

Since $\forall x \in \mathbb{R},\left(x, 1, e^x\right) \in \mathcal{K}$, we know $\forall(u, v, w) \in K^*$,
$$
f(x)=u x+v+w e^x \geq 0, \forall x \in \mathbb{R} .
$$


If $u>0$, then when $x \rightarrow -\infty, u x \rightarrow-\infty, w_e^x \rightarrow 0 \Rightarrow f(x) \rightarrow-\infty$, it leads to a contradiction! Therefore $u \leq 0$. 

Note that $f^{\prime}(x)=w e^x+u$.

If $u=0$, then $f(x)$ increases when $x$ increases, i.e., $\inf_{x \in \real} f(x)=\lim_{x \rightarrow \infty} f(x)=v$. Therefore, $(0, v, w) \in K^*$ for $w>0$ if $v \geq 0$.

If $u<0$, then $f(x)$ is monotonically increasing for $x \in\left(-\infty, \log \frac{-u}{w}\right)$ and monotonically decreasing for $x \in\left(\log \frac{-u}{w},+\infty\right)$. 

Therefore 
$$\min_{x \in \mathbb{R}} f(x)=f\left(\log \frac{-u}{w}\right)=u \log \frac{-u}{w}+v+w \cdot\left(\frac{-u}{w}\right)=u\left(\log \frac{-u}{w}-1\right)+v$$

Consequently, $(u, v, w) \in K^*$ for $u<0, w>0$ if $v \geq u\left(1-\log \frac{-u}{w}\right)$.

In conclusion, 
\[
K^* = \{(u, v, w) \mid  u = 0, v, w \geq 0\} \cup \{(u, v, w) \mid  u<0, w>0, v\geq u(1-\log \frac{-u}{w})\}
\]

\end{solution}



\clearpage
\section*{Problem 2}
Consider the following optimization problem:
\begin{align}
    \min_{x\in\real^n}\quad &f(x)\notag\\
    \text{s.t.}\quad 
    &\ones\T x = 1\tag{P1}\\
    &x\succeq 0\notag
\end{align}
where $\ones$ is a vector of ones, i.e. $\ones = [1, \dots, 1]\T\in\real^n$. Assume that $f(x)$ is convex and differentiable.

(a) Prove that $x^*$ is optimal for (P1) if and only if $x^*$ is feasible and
\[
\min_{1\leq i \leq n} \grad f(x^*)_i \geq \grad f(x^*)\T x^*
\]
where $\grad f(x^*)_i$ is the ith element of $\grad f(x^*)$.

(b) Prove that $x^*$ is optimal for (P1) if and only if $x^*$ is feasible and, for each $k$,
\[
x_k^* > 0 \Rightarrow \grad f(x^*)_k = \min_{1\leq i\leq n}\grad f(x^*)_i
\]

\begin{proof}
Since $f(x)$ is convex, by the first order optimality condition, $x$ is optimal if and only if $x^*$ is feasible and for any feasible $y\in\dom f$,
\[
\grad f(x^*)\T (y - x^*) \geq 0 \Rightarrow \grad f(x^*)\T y \geq  \grad f(x^*)\T x^*
\]

(a) 

"$\Rightarrow$": 
If $x^*$ is optimal, for any $i = 1, \dots, n$, let $e_i$ be the unit vector such that all elements except the $i$-th element are 0 , and the $i$-th element is 1.

Therefore, $\ones^{\top} e_i=1$ and $e_i \geq 0$, which indicates that $e_i$ is feasible\, for any $i = 1, \dots, n$. 

Then $\forall i = 1, \dots, n$, we have
$$
\begin{aligned}
& \nabla f\left(x^*\right)^{\top}\left(e_i-x^*\right) \geq 0 \\
\Leftrightarrow\quad & \nabla f\left(x^*\right)^{\top} e_i \geq \nabla f\left(x^*\right)^{\top} x^*\\
\Leftrightarrow\quad & \nabla f\left(x^*\right)_i \geq \nabla f\left(x^*\right)^{\top} x^*\\
\Leftrightarrow\quad & \min _{1 \leq i \leq n} \nabla f\left(x^*\right)_i \geq \nabla f\left(x^*\right)^{\top} x^*
\end{aligned}
$$

"$\Leftarrow$":
If $\min_{1 \leq i \leq n} \nabla f\left(x^*\right)_i \geq \nabla f\left(x^*\right)^{\top} x^*$, then for any $y\succeq 0, \ones\T y = 1$,
\begin{align*}
    \grad f(x^*)\T y
    & = \sum_{i=1}^{n} y_i \grad f(x^*)_i\\
    & \geq \min_{1 \leq i \leq n} \grad f(x^*)_i \sum_{i=1}^{n} y_i \\
    & = \min_{1 \leq i \leq n} \grad f(x^*)_i \ones\T y\\
    & = \min_{1 \leq i \leq n} \grad f(x^*)_i\\
    & \geq \grad f(x^*)\T x^*
\end{align*}
which implies $\grad f(x^*)\T y \geq \grad f(x^*)\T x^*$ for any feasible $y$. Thus, $x^*$ is optimal.

(b) From the result of part (a), we only need to prove that 
$$\min_{1\leq i \leq n} \grad f(x^*)_i \geq \grad f(x^*)\T x^*$$
if and only if $x^*$ is feasible, and for each $k = 1, \dots, n$, 
$$x_k^* > 0 \Rightarrow \grad f(x^*)_k = \min_{1\leq i\leq n}\grad f(x^*)_i$$


"$\Rightarrow$": 
If $\min_{1 \leq i \leq n} \nabla f\left(x^*\right)_i \geq \nabla f\left(x^*\right)^{\top} x^*, x^*$ is feasible and $x_k^*>0$ for each $k = 1, \dots, n$, then we have
$$
\begin{aligned}
\nabla f\left(x^*\right)^{\top} x^* & =\nabla f\left(x^*\right)_k  x_k^*+\sum_{i \neq k} \nabla f\left(x^*\right)_i x_i^* \\
& \geq \nabla f\left(x^*\right)_k  x_k^*+\sum_{i \neq k} x_i^*  \min _{1 \leq i \leq n} \nabla f\left(x^*\right)_i \\
& =\nabla f\left(x^*\right)_k  x_k^*+\left(1-x_k^*\right) \min _{1 \leq i \leq n} \nabla f\left(x^*\right)_i
\end{aligned}
$$


Given that $\min_{1 \leq i \leq n} \nabla f\left(x^*\right)_i \geq \nabla f\left(x^*\right)^{\top} x^*$, and $x_k^*>0$, we have
$$
\begin{aligned}
& \min _{1 \leq i \leq n} \nabla f\left(x^*\right)_i-\left(1-x_k^*\right)\min_{1 \leq i \leq n} \nabla f\left(x^*\right)_i \geq x_k^* \nabla f\left(x^*\right)_k \\
\Leftrightarrow \quad
& \min _{1 \leq i \leq n} \nabla f\left(x^*\right)_i \geq \nabla f\left(x^*\right)_k .
\end{aligned}
$$


Since we also have $\nabla f\left(x^*\right)_k \geq \min _{1 \leq i \leq n} \nabla f\left(x^*\right)_i$, we must have $\nabla f\left(x^*\right)_k=\min _{i \leq i \leq n} \nabla f\left(x^*\right)_i$.


"$\Leftarrow$":
If $x^*$ is feasible and for each $k$, $x_k^*>0 \Rightarrow \nabla f\left(x^*\right)_k=\min _{1\leq  i \leq n} \nabla f\left(x^*\right)_i$, we can separate $x$ as two parts. First part we have $x_k$ for some $k$ such that $x_k > 0$, and the second part we have $x_l$ for $l$ includes the rest of $x$ such that $x_l = 0$.
$$
\begin{aligned}
\nabla f\left(x^*\right)^{\top} x^* & =\sum_{i=1}^n x_i^* \nabla f\left(x^*\right)_i \\
& =\sum_{k} x_k^* \nabla f\left(x^*\right)_k+\sum_{l} x_l^* \nabla f\left(x^*\right)_l \\
% & =\sum_{k} x_k^* \min _{1 \leq i \leq n} \nabla f\left(x^*\right)_i \\
& =\min _{1 \leq i \leq n} \nabla f\left(x^*\right)_i \sum_{k} x_k^* \\
& =\min _{l \leq i \leq n} \nabla f\left(x^*\right)_i(\ones^{\top} x^*-\sum_{l} x_l^*) \\
& =\min _{1 \leq i \leq n} \nabla f\left(x^*\right)_i
\end{aligned}
$$


Therefore, $\nabla f\left(x^*\right)^{\top} x^* \leq \min _{1\leq i \leq n} \nabla f\left(x^*\right)_i$.
\end{proof}

\clearpage
\section*{Problem 3}
Consider the function $f:\real^{n \times n}\times \real^n \rightarrow \real$ defined by $f(X, y) = y\T X^{-1}y$, where $X\in \real^{n\times n}$ and $y\in\real^n$.

(a) Suppose that $\dom f=\{(X, y) : X\in\symm^n_{++}\}$. Is $f(X, y)$ convex? If so, prove it. If not, give a counter-example.

(b) Suppose that $\dom f=\{(X, y): X+X\T \in\symm^n_{++}\}$. Is $f(X, y)$ convex? If so, prove it. If not, give a counter-example.


\begin{solution}
    (a)
    To determine if \( f(X, y) \) is convex, we can check if the epigraph of \( f \), defined as
    \[
    \epi f = \{ (X, y, t) \mid y\T X^{-1} y \leq t, \; X \succ 0 \},
    \]
    is convex. Using the Schur complement, we have
    \[
    t - y\T X^{-1} y \geq 0 \quad \text{and} \quad X \succ 0 \quad \Longleftrightarrow \quad \begin{bmatrix} X & y \\ y\T & t \end{bmatrix} \succ 0, X\succ 0 
    \]
    Thus, the epigraph can be expressed as
    \[
    \epi f = \{ (X, y, t) \mid \begin{bmatrix} X & y \\ y\T & t \end{bmatrix} \succ 0, \; X \succ 0 \}.
    \]
    This set is the intersection of two convex sets (a positive semi-definite matrix constraint and \( X \succ 0 \)), which is itself convex. 
    
    Therefore, \( f(X, y) \) is convex when \( \dom f = \{(X, y) \mid X \in \mathbb{S}_{++}^n\} \).
    
    (b)
    When the domain is changed to \( \dom f = \{(X, y) \mid X + X\T \in \mathbb{S}_{++}^n \} \), the function \( f(X, y) \) is no longer convex. We can find a counterexample which violates the convexity.
    
    Consider the matrices and vectors:
    \[
    X_1 = \begin{bmatrix} 2 & 1 \\ 1 & 2 \end{bmatrix}, \quad X_2 = \begin{bmatrix} 2 & -1 \\ -1 & 2 \end{bmatrix}, \quad y_1 = \begin{bmatrix} 1 \\ 0 \end{bmatrix}, \quad y_2 = \begin{bmatrix} 0 \\ 1 \end{bmatrix}.
    \]
    
    Evaluate it on the Jensen's inequality.
    \[ f( \half (X_1 + X_2), \half(y_1 + y_2) ) = \frac{1}{4} \]
    and
    \[
        \half f(X_1, y_1) + \half f(X_2, y_2) = \frac{2}{3}
    \]
    which does not satisfy convexity in this domain.    
\end{solution}


\clearpage
\section*{Problem 4}
Let $f:\real^n\rightarrow \real$ be a convex function with $f(0)\leq 0$.

(a) Show that the perspective function $tf(x/t)$ with domain $\{(x, t) \mid  t>0, x/t\in\dom f\}$ is nonincreasing as a function of $t$ for a fixed $x$.

(b) Let $g:\real^n\rightarrow \real$ be concave and positive on its domain. Show that the following function is convex:
\[
h(x) = g(x)f(x/g(x)), \quad \dom h=\{x\in\dom g \mid  x/g(x) \in \dom f\}
\]

(c) Using the result of part (b), show that the following function is convex:
\[
h(x) = \frac{\|x\|_2^2}{(\prod_{i=1}^{n}x_i)^{1/n}},\quad \dom h = \real_{++}^n
\]


\begin{proof}
(a) 
Take $t$ and $s$, such that $0<t\leq s$ and $x/t, x/s \in \dom f$,
\begin{align*}
    f(\frac{x}{s}) &= f(\frac{x}{t}\cdot\frac{t}{s} + 0\cdot (1-\frac{t}{s}))\\
    & \leq \frac{t}{s}f(\frac{x}{t}) + (1-\frac{t}{s}) f(0)\\
    & \leq \frac{t}{s}f(\frac{x}{t})\\
    \Leftrightarrow \quad
    sf(\frac{x}{s}) & \leq tf(\frac{x}{t})
\end{align*}
which implies $tf(x/t)$ is nonincreasing for a fixed $x$.

(b)
To show $h(x)$ is convex, we first show $\dom h$ is convex, i.e. $\forall x, y \in\dom h, \alpha \in [0, 1] \Rightarrow \alpha x + (1-\alpha)y \in\dom h$, which is equivalent to show that
\[
\alpha x + (1-\alpha) y \in \dom g \ \text{ and }\  \frac{\alpha x + (1-\alpha) y}{g(\alpha x + (1-\alpha) y)} \in\dom f
\]

Take $\forall x, y \in\dom h, \alpha \in [0, 1]$. 

Given $g(x)$ is concave, we have $\dom g$ is convex.

Therefore, $\alpha x + (1-\alpha) y \in \dom g$.

\begin{align*}
    \frac{\alpha x + (1-\alpha) y}{g(\alpha x + (1-\alpha) y)}
    =& \frac{\alpha g(x)}{g(\alpha x + (1-\alpha) y)} \cdot \frac{x}{g(x)} \\
    &+ \frac{(1-\alpha)g(y)}{g(\alpha x + (1-\alpha) y)}\cdot \frac{y}{g(y)} \\
    &+ (1 - \frac{\alpha g(x) + (1-\alpha)g(y)}{g(\alpha x + (1-\alpha) y)}) \cdot 0
\end{align*}
where $x/g(x), y/g(y), 0\in\dom f$ and their corresponding coefficients sum up to $1$.

This implies it is a convex combination of elements within a convex set $\dom f$, which also lies in the same set, i.e.
\[
    \frac{\alpha x + (1-\alpha) y}{g(\alpha x + (1-\alpha) y)} \in\dom f
\]

Then, we need to show that $h(x)$ satisfies the Jensen's inequality, i.e. 
\[
    h(\alpha x + (1-\alpha) y) \leq \alpha h(x) + (1-\alpha) h(y)
\]

\begin{align*}
    h(\alpha x + (1-\alpha) y) 
    & = g(\alpha x + (1-\alpha) y) f(\frac{\alpha x + (1-\alpha) y}{g(\alpha x + (1-\alpha) y)})\\
    & \leq (\alpha g(x) + (1-\alpha)g(y)) f(\frac{\alpha x + (1-\alpha) y}{\alpha g(x) + (1-\alpha)g(y)}) \Leftarrow  \text{from part (a)}\\
    & = (\alpha g(x) + (1-\alpha)g(y)) f(\frac{\alpha g(x)}{\alpha g(x) + (1-\alpha)g(y)} \cdot \frac{x}{g(x)} + \frac{(1-\alpha) g(y)}{\alpha g(x) + (1-\alpha)g(y)} \cdot \frac{y}{g(y)})\\
    & \leq \alpha g(x) f( \frac{x}{g(x)}) + (1-\alpha) g(y) f(\frac{y}{g(y)}) \Leftarrow \text{$f$ is convex}\\
    & = \alpha h(x) + (1-\alpha) h(y)
\end{align*}

Thus, $h(x)$ is convex.

(c)
Take $f(x) = \|x\|_2^2$, such that $f(x)$ is convex and $f(0) = 0$.
Take $g(x) = (\prod_{i=1}^{n})^{1/n}, \dom g = \real^n_{++}$, such that $g(x)$ is concave and positive on its domain.

From the results of part (a) and (b), the function $h(x)$
\begin{align*}
    h(x) & = g(x) f(x/g(x)) \\
    & = (\prod_{i=1}^{n})^{1/n} \left\|\frac{x}{(\prod_{i=1}^{n})^{1/n}}\right\|_2^2\\
    & = \frac{\|x\|_2^2}{(\prod_{i=1}^{n}x_i)^{1/n}}
\end{align*}
with domain
\[
\dom h = \{x\in\dom g, x/g(x) \in \dom f\} = \real^n_{++}
\]
is a convex function.

\end{proof}


\clearpage
\section*{Problem 5}
(a) Let $X = \{x_1, x_2, \dots, x_m\}$ be a set of points in $\real^n$, where $m\geq n+2$. Show that $X$ can be partitioned in two sets $Y$ and $Z = X\backslash Y$ such that $\Co Y \cap \Co Z$ is non-empty. Here, $\Co Y$ and $\Co Z$ refer to the convex hulls of $Y$ and $Z$, respectively.

(b) Let $S_1, \dots, S_m$ be a collection of convex sets in $\real^n$, where $m\geq n + 2$. Suppose that the intersection of every $m-1$ sets from this collection is non-empty. Then the intersection of all sets $S_1\cap\dots\cap S_m$ is non-empty.


\begin{proof}
    (a) Since $\left[x_i, 1\right]^{\top}, 1 \leq i \leq m$ are $\mathbb{R}^{n+1}$ vectors and $m>n+1$, we know that they are linearly dependent, i.e. $\exists y \in \mathbb{R}^m \backslash\{0\}$,
    $$
    \left[\begin{array}{ccc}
    x_1 & \cdots & x_m \\
    1 & \cdots & 1
    \end{array}\right]\left[\begin{array}{c}
    y_1 \\
    \vdots \\
    y_m
    \end{array}\right]=0 
    \quad\Rightarrow\quad
    \begin{cases}
        \sum_{i=1}^m y_i x_i=0 \\
        \sum_{i=1}^m y_i=0 .
    \end{cases}
    $$
    
    Define $S=\left\{i\mid i = 1, \dots,  m, y_i \geq 0\right\},\ T=\{i\mid i=1, \dots, m\} \backslash S=\left\{i\mid i = 1, \dots,  m, y_i<0\right\}$. Since $y \neq 0$ we must have $S, T \neq \varnothing$. 
    
    Therefore, we have
    \begin{align}
        \begin{cases}  
            \sum_{i=1}^m y_i x_i=0 &\Leftrightarrow\quad \sum_{i \in S} y_i x_i=\sum_{j \in T}\left(-y_j\right) x_j\\
            \sum_{i=1}^m y_i=0 &\Leftrightarrow\quad \sum_{i \in S} y_i=\sum_{j \in T}\left(-y_j\right)>0 \\
            \forall i \in S, y_i \geq 0 ; \forall j \in T,-y_j>0
        \end{cases}
        \label{eq:p5}
    \end{align}
    
    Define $c=\sum_{i \in S} y_i>0$ and
    \[
    z_{i|j} = 
    \begin{cases}
        y_i / c & \text{if } i \in S\\
        -y_j / c & \text{if }  j \in T
    \end{cases}
    \]

    Then, divide \eqref{eq:p5} by $c$, we have
    \begin{align*}
        &\sum_{i \in S} z_i x_i=\sum_{j \in T} z_j x_j \coloneqq v\\
        &\sum_{i \epsilon s} z_i=\sum_{j \in T} z_j=1\\
        &\forall i \in\{1, \cdots, m\}, z_i \geq 0
    \end{align*}
    which implies that $v$ can be expressed as two convex combinations of $z_{i|j}, x_{i|j}$ for $i\in S, j\in T$.
    
    Consequently, let $Y=\left\{x_i\mid i \in S\right\}, Z=\left\{x_j\mid j \in T\right\}$, we have $\exists v \in \mathbb{R}^n, \ v\in\Co Y$ and $v\in\Co  Z$, i.e.
    $\Co Y \cap \Co Z$ is non-empty.
\end{proof}


\end{document}
