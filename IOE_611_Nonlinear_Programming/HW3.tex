\documentclass[11pt]{article}
\usepackage{geometry}                
\geometry{letterpaper,tmargin=1in,bmargin=1in,lmargin=1in,rmargin=1in} 
\usepackage[parfill]{parskip}    % Activate to begin paragraphs with an empty line rather than an indent
\usepackage{graphicx}
\usepackage{amsmath, amsfonts, amsthm, amssymb} 
\usepackage[shortlabels]{enumitem}

\parskip = 0.1in

\pagestyle{myheadings}
\markright{Homework of IOE 611 Nonlinear Programming\hfill Yulun Zhuang \hfill}

 % some traditional definitions that can be blamed on craig barratt
 \newcommand{\BEAS}{\begin{eqnarray*}}
\newcommand{\EEAS}{\end{eqnarray*}}
\newcommand{\BEA}{\begin{eqnarray}}
\newcommand{\EEA}{\end{eqnarray}}
\newcommand{\BEQ}{\begin{equation}}
\newcommand{\EEQ}{\end{equation}}
\newcommand{\BIT}{\begin{itemize}}
\newcommand{\EIT}{\end{itemize}}

% text abbrevs
\newcommand{\eg}{e.g.}
\newcommand{\ie}{i.e.}

% std math stuff
\newcommand{\ones}{\mathbf 1}
\newcommand{\real}{{{\mathbb{R}}}}
\newcommand{\integer}{{{\mathbb{Z}}}}
\newcommand{\complex}{{{\mathbb{C}}}}
\newcommand{\symm}{{{\mathbb{S}}}}  % symmetric matrices
%
% lin alg stuff
\newcommand{\Span}{\mbox{\textrm{span}}}
\newcommand{\range}{{\mathcal R}}
\newcommand{\nullspace}{{\mathcal N}}
\newcommand{\Rank}{\mathop{\textbf{rank}}}
\newcommand{\Tr}{\mathop{\textbf{tr}}}
\newcommand{\cond}{\mathop{\textbf{cond}}}
\newcommand{\diag}{\mathop{\textbf{diag}}}
\newcommand{\lambdamax}{\lambda_{\rm max}}
\newcommand{\lambdamin}{\lambda_{\rm min}}

% probability stuff
\newcommand{\Prob}{\mathop{\textbf{prob}}}
\newcommand{\Expect}{\mathop{\textbf{E{}}}}
\newcommand{\var}{\mathop{\textbf{var}}} % variance
% not sure why we have \Expect and \Prob but \var ???

% convexity & optimization stuff
\newcommand{\Co}{\mathop {\textbf{conv}}} % convex hull
\newcommand{\argmin}{\mathop{\rm argmin}}
\newcommand{\argmax}{\mathop{\rm argmax}}
\newcommand{\epi}{\mathop{\textbf{epi}}}
%\newcommand{\hypo}{\mathop{\textbf{hypo}}}}

% sup and inf that look OK in saddle-point form!
%\newcommand{\ourinf}{\mathop{\raisebox{0ex}[0ex][.4ex]{\,inf\,}}}
%\newcommand{\oursup}{\mathop{\raisebox{0ex}[0ex][.4ex]{\,sup\,}}}
\newcommand{\ourinf}{\mathop{\,\mathrm{inf}\, {\rule[-.5ex]{0ex}{0ex}}}}
\newcommand{\oursup}{\mathop{\,\mathrm{sup}\, {\rule[-.5ex]{0ex}{0ex}}}}
%makes latex believe that inf and sup both extend .4ex below
%the baseline

\newcommand{\dist}{\mathop{\textbf{dist}}}
\newcommand{\vol}{\mathop{\textbf{vol}}} % volume
\newcommand{\Card}{\mathop{\textbf{card}}} % cardinality
\newcommand{\sign}{\mathop{\textbf{sign}}}

\newcommand{\dom}{\mathop{\textbf{dom}}} % domain
\newcommand{\aff}{\mathop{\textbf{aff}}} % affine hull
\newcommand{\cl}{\mathop{\textbf{cl}}} % closure
\newcommand{\intr}{\mathop{\textbf{int}}} % interior
\newcommand{\relint}{\mathop{\textbf{rel int}}} % relative interior
\newcommand{\bd}{\mathop{\textbf{bd}}} % boundary

%why do we have the following but not \nust?
\newcommand{\xst}{x^\star}
\newcommand{\lambdast}{\lambda^\star}

% defs for cones & generalized inequalities
% these seem kind of awkward; should fix some day
% rewrite them to use args?
\newcommand{\geqK}{\mathrel{\succeq_K}}
\newcommand{\gK}{\mathrel{\succ_K}}
\newcommand{\leqK}{\mathrel{\preceq_K}}
\newcommand{\lK}{\mathrel{\prec_K}}
\newcommand{\geqKst}{\mathrel{\succeq_{K^*}}}
\newcommand{\gKst}{\mathrel{\succ_{K^*}}}
\newcommand{\leqKst}{\mathrel{\preceq_{K^*}}}
\newcommand{\lKst}{\mathrel{\prec_{K^*}}}
\newcommand{\geqL}{\mathrel{\succeq_L}}
\newcommand{\gL}{\mathrel{\succ_L}}
\newcommand{\leqL}{\mathrel{\preceq_L}}
\newcommand{\lL}{\mathrel{\prec_L}}
\newcommand{\geqLst}{\mathrel{\succeq_{L^*}}}
\newcommand{\gLst}{\mathrel{\succ_{L^*}}}
\newcommand{\leqLst}{\mathrel{\preceq_{L^*}}}
\newcommand{\lLst}{\mathrel{\prec_{L^*}}}

%\newcounter{lecture}
%\newcommand{\lecturefl}[1]{   % use with foiltex landscape
%% \addtocounter{lecture}{1}
% \refstepcounter{lecture}
% \setcounter{equation}{0}
% \setcounter{page}{1}
% \renewcommand{\theequation}{\arabic{equation}}
% \renewcommand{\thepage}{\arabic{lecture}--\arabic{page}}
% \raggedright
% \parindent 0pt
% \rightfooter{\thepage}
% \leftheader{}
% \rightheader{}
% \LogoOff
% \input header 
% \begin{center}
%% {\Large \bfseries Lecture \arabic{lecture} \\*[\bigskipamount] {#1}}
%{\Large \bfseries \arabic{lecture}.  {#1}}
% \end{center}
% \MyLogo{#1}
%}

%\newcommand{\lectureflstar}[1]{   % use with foiltex landscape
% \setcounter{equation}{0}
% \setcounter{page}{1}
% \renewcommand{\theequation}{\arabic{equation}}
% \renewcommand{\thepage}{\arabic{page}}
% \raggedright
% \parindent 0pt
% \rightfooter{\thepage}
% \leftheader{}
% \rightheader{}
% \LogoOff
% \input header 
% \begin{center}
% {\Large \bfseries #1}
% \end{center}
% \MyLogo{#1}
%}
%\newcounter{oursection}
%\newcommand{\frametitle}[1]{  % for use with foiltex landscape
% \addtocounter{oursection}{1}
%% \setcounter{equation}{0}
% \foilhead[-1.0cm]{#1}
% \LogoOn
%}

\newenvironment{algdesc}%
   {\begin{list}{}{%
   \setlength{\rightmargin}{0\linewidth}%
   \setlength{\leftmargin}{.05\linewidth}}%
   \sffamily\small
   \item[]{\setlength{\parskip}{0ex}\hrulefill\par%
   \nopagebreak{}}}%
   {{\setlength{\parskip}{-1ex}\nopagebreak\par\hrulefill} \end{list}}

\newenvironment{colm}{\left[\begin{array}{c}}{\end{array}\right]}
\newenvironment{colv}{\left(\begin{array}{c}}{\end{array}\right)}


\newcommand{\oh}{\frac12}
\newcommand{\st}{\text{subject to}}
\newcommand{\gfb}{\nabla f(\bar x)}
\newcommand{\hfb}{H(\bar x)}

\newtheorem{theorem}{Theorem}[section]
\newtheorem{remark}[theorem]{Remark}%[section]
\newtheorem{definition}[theorem]{Definition}%[section]
\newtheorem{proposition}[theorem]{Proposition}%[section]
\newtheorem{lemma}[theorem]{Lemma}%[section]
\newtheorem{corollary}[theorem]{Corollary}%[section]
\newtheorem{assumption}{Assumption}
\newtheorem{claim}{Claim}
\newtheorem{exam}{Example}
\newenvironment{solution}
  {\renewcommand\qedsymbol{$\square$}\begin{proof}[Solution]}
  {\end{proof}}


\newcommand{\red}[1]{\textcolor{red}{#1}}
\newcommand{\blue}[1]{\textcolor{blue}{#1}}
\newcommand{\green}[1]{\textcolor{green}{#1}}
\newcommand{\grad}{\nabla}
\newcommand{\hess}{\nabla^2}
\newcommand{\tr}{\text{tr}}

\newcommand{\dd}{\mathrm{d}}
\newcommand{\RR}{\mathbb{R}}
\newcommand{\NN}{\mathbb{N}}
\newcommand{\ZZ}{\mathbb{Z}}
\newcommand{\bS}{\mathbb{S}}

\newcommand{\pd}[2][]{ \frac{\partial #1}{\partial #2}} % Partial derivatives
\renewcommand{\d}{{\rm d}}
\newcommand{\ddt}{\frac{\d}{\d t}}
\newcommand{\half}{\frac{1}{2}}
\newcommand{\inv}{^{-1}}
\newcommand{\T}{^\top}

\begin{document}
\title{IOE 611: Homework 3}
\author{Yulun Zhuang}
\maketitle
%**********************************
\section*{Problem 1}
Prove that $x^* = (1, 1/2, -1)$ is optimal for the optimization problem
\begin{align*}
  &\begin{array}{ll}
  \text{minimize} & (1 / 2) x^T P x+q^T x+r \\
  \text {subject to} & -1 \leq x_i \leq 1, \quad i=1,2,3
  \end{array}\\
\end{align*}
where
\begin{align*}
  &P=\left[\begin{array}{rrr}
  13 & 12 & -2 \\
  12 & 17 & 6 \\
  -2 & 6 & 12
  \end{array}\right], \quad q=\left[\begin{array}{r}
  -22.0 \\
  -14.5 \\
  13.0
  \end{array}\right], \quad r=1
\end{align*}

\begin{proof}
  The gradient of the objective function at $x^*$ is
  \begin{align*}
    &\grad f_0(x) = P x + q\\
    &\grad f_0(x^*) = [-1, 0, 2]\T
  \end{align*}

  The optimality condition is
  \begin{align*}
      \nabla f_0(x^*)\T (y - x^*)
      &= (-1) (y_1 - 1) + 2 (y_3 + 1) \\
      &= -y_1 + 2y_3 + 3
  \end{align*}
  which is always non-negative for the feasible region $\{ y_i \mid -1 \leq y_i \leq 1 \}$.
\end{proof}




\clearpage
\section*{Problem 2}
\textit{Equivalent convex problems}.  Show that the following three convex problems are equivalent.
Carefully explain how the solution of each problem is obtained from the solution of
the other problems. The problem data are the matrix $A \in \RR^{m\times n}$ (with rows $a\T_i$), the vector $b \in\RR^m$, and the constant $M > 0$.

(a) The \textit{robust least-squares problem}
\[
\text{minimize}\quad \sum_{i=1}{m} \Phi(a\T_i x - b_i)
\]
with variable $x\in\RR^n$, where $\Phi:\RR \rightarrow \RR$ is define as
\[
  \Phi(u)= 
  \begin{cases}
    u^2, & |u| \leq M \\ 
    M (2|u| - M), & |u| > M
  \end{cases}
\]

(b) The \textit{least-squares problem with variable weights}
\begin{align*}
  &\text{minimize}\quad \sum_{i=1}^{m} (a\T_i x - b_i)^2 / (w_i + 1) + M^2 \mathbf{1}\T w\\
  &\text{subject to}\quad w\succeq 0
\end{align*}
with variables $x\in\RR^n$ and $w\in\RR^m$, and domain $D = \{(x, w) \in\RR^n\times \RR^m | w\succ -1\}$.


(c) The \textit{quadratic program}
\begin{align*}
  \text{minimize}\quad &\sum_{i=1}^{m}(u_i^2 + 2Mv_i)\\
  \text{subject to}\quad &-u -v \preceq Ax - b \preceq u+v\\
  & 0 \preceq u \preceq M \mathbf{1}\\
  & v \succeq 0
\end{align*}


\clearpage
\section*{Problem 3}
\textit{Robust LP with interval coefficients}. Consider the problem, with variable $x\in\RR^n$,
\begin{align*}
  \text{minimize}\quad &c\T x\\
  \text{subject to}\quad & Ax\preceq b, \ \forall A\in \mathcal{A}
\end{align*}
where $\mathcal{A} \subseteq \RR^{m\times n}$ is the set
\[
\mathcal{A} = \{A\in \RR^{m\times n} | \bar A_{ij} - V_{ij} \leq A_{ij} \leq \bar A_{ij} + V_{ij}, i = 1, \dots, m, j = 1, \dots, n\}
\]
(The matrices $\bar A$ and $V$ are given.) This problem can be interpreted as an LP where each coefficient of $A$ is only known to lie in an interval, and we require that $x$ must satisfy the constraints for all possible values of the coefficients. 

Express this problem as an LP. The LP you construct should be efficient, i.e., it should not have dimensions that grow exponentially with $n$ or $m$.

% TODO add 8.16


\clearpage
\section*{Problem 4}
\textit{Minimum fuel optimal control}. We consider a linear dynamical system with state $x(t)\in \RR^n, t=0, \dots, N$, and actuator or input signal $u(t)\in\RR, t=0, \dots, N-1$. The dynamics of the system is given by the linear recurrence
\[
x(t+1) = Ax(t) + bu(t),\ t=0, \dots, N-1
\]
where $A\in\RR^{n\times n}$ and $b\in \RR$ are given. We assume that the initial state is zero, i.e., $x(0) = 0$.

The \textit{minimum fuel optimal control problem} is to choose the inputs $u(0), \dots, u(N-1)$ so as to minimize the total fuel consumed, which is given by
\[
F = \sum_{t=0}^{N-1} f(u(t))
\]
subject to the constraint that $x(N) = x_{\text{des}}$, where $N$ is the (given) time horizon, and $x_{\text{des}}$ is the (given) desired final or target state. The function $f:\RR\rightarrow \RR$ is the fuel use map for the actuator, and gives the amount of fuel used as a function of the actuator signal amplitude. In this problem we use
\[
f(a) = 
\begin{cases}
  |a| & |a| \leq 1\\
  2|a| -1 & |a| >1
\end{cases}
\]
This means that fuel use is proportional to the absolute value of the actuator signal, for actuator signals between -1 and 1; for larger actuator signals the marginal fuel efficiency is half.

Formulate the minimum fuel optimal control problem as an LP.

\end{document}
