\documentclass[11pt]{article}
\usepackage{geometry}                
\geometry{letterpaper,tmargin=1in,bmargin=1in,lmargin=1in,rmargin=1in} 
\usepackage[parfill]{parskip}    % Activate to begin paragraphs with an empty line rather than an indent
\usepackage{graphicx}
\usepackage{amsmath, amsfonts, amsthm, amssymb} 
\usepackage[shortlabels]{enumitem}
\usepackage{xcolor}
\usepackage{mathtools}

\parskip = 0.1in

\pagestyle{myheadings}
\markright{Homework of IOE 611 Nonlinear Programming\hfill Yulun Zhuang \hfill}

 % some traditional definitions that can be blamed on craig barratt
 \newcommand{\BEAS}{\begin{eqnarray*}}
\newcommand{\EEAS}{\end{eqnarray*}}
\newcommand{\BEA}{\begin{eqnarray}}
\newcommand{\EEA}{\end{eqnarray}}
\newcommand{\BEQ}{\begin{equation}}
\newcommand{\EEQ}{\end{equation}}
\newcommand{\BIT}{\begin{itemize}}
\newcommand{\EIT}{\end{itemize}}

% text abbrevs
\newcommand{\eg}{e.g.}
\newcommand{\ie}{i.e.}

% std math stuff
\newcommand{\ones}{\mathbf 1}
\newcommand{\real}{{{\mathbb{R}}}}
\newcommand{\integer}{{{\mathbb{Z}}}}
\newcommand{\complex}{{{\mathbb{C}}}}
\newcommand{\symm}{{{\mathbb{S}}}}  % symmetric matrices
%
% lin alg stuff
\newcommand{\Span}{\mbox{\textrm{span}}}
\newcommand{\range}{{\mathcal R}}
\newcommand{\nullspace}{{\mathcal N}}
\newcommand{\Rank}{\mathop{\textbf{rank}}}
\newcommand{\Tr}{\mathop{\textbf{tr}}}
\newcommand{\cond}{\mathop{\textbf{cond}}}
\newcommand{\diag}{\mathop{\textbf{diag}}}
\newcommand{\lambdamax}{\lambda_{\rm max}}
\newcommand{\lambdamin}{\lambda_{\rm min}}

% probability stuff
\newcommand{\Prob}{\mathop{\textbf{prob}}}
\newcommand{\Expect}{\mathop{\textbf{E{}}}}
\newcommand{\var}{\mathop{\textbf{var}}} % variance
% not sure why we have \Expect and \Prob but \var ???

% convexity & optimization stuff
\newcommand{\Co}{\mathop {\textbf{conv}}} % convex hull
\newcommand{\argmin}{\mathop{\rm argmin}}
\newcommand{\argmax}{\mathop{\rm argmax}}
\newcommand{\epi}{\mathop{\textbf{epi}}}
%\newcommand{\hypo}{\mathop{\textbf{hypo}}}}

% sup and inf that look OK in saddle-point form!
%\newcommand{\ourinf}{\mathop{\raisebox{0ex}[0ex][.4ex]{\,inf\,}}}
%\newcommand{\oursup}{\mathop{\raisebox{0ex}[0ex][.4ex]{\,sup\,}}}
\newcommand{\ourinf}{\mathop{\,\mathrm{inf}\, {\rule[-.5ex]{0ex}{0ex}}}}
\newcommand{\oursup}{\mathop{\,\mathrm{sup}\, {\rule[-.5ex]{0ex}{0ex}}}}
%makes latex believe that inf and sup both extend .4ex below
%the baseline

\newcommand{\dist}{\mathop{\textbf{dist}}}
\newcommand{\vol}{\mathop{\textbf{vol}}} % volume
\newcommand{\Card}{\mathop{\textbf{card}}} % cardinality
\newcommand{\sign}{\mathop{\textbf{sign}}}

\newcommand{\dom}{\mathop{\textbf{dom}}} % domain
\newcommand{\aff}{\mathop{\textbf{aff}}} % affine hull
\newcommand{\cl}{\mathop{\textbf{cl}}} % closure
\newcommand{\intr}{\mathop{\textbf{int}}} % interior
\newcommand{\relint}{\mathop{\textbf{rel int}}} % relative interior
\newcommand{\bd}{\mathop{\textbf{bd}}} % boundary

%why do we have the following but not \nust?
\newcommand{\xst}{x^\star}
\newcommand{\lambdast}{\lambda^\star}

% defs for cones & generalized inequalities
% these seem kind of awkward; should fix some day
% rewrite them to use args?
\newcommand{\geqK}{\mathrel{\succeq_K}}
\newcommand{\gK}{\mathrel{\succ_K}}
\newcommand{\leqK}{\mathrel{\preceq_K}}
\newcommand{\lK}{\mathrel{\prec_K}}
\newcommand{\geqKst}{\mathrel{\succeq_{K^*}}}
\newcommand{\gKst}{\mathrel{\succ_{K^*}}}
\newcommand{\leqKst}{\mathrel{\preceq_{K^*}}}
\newcommand{\lKst}{\mathrel{\prec_{K^*}}}
\newcommand{\geqL}{\mathrel{\succeq_L}}
\newcommand{\gL}{\mathrel{\succ_L}}
\newcommand{\leqL}{\mathrel{\preceq_L}}
\newcommand{\lL}{\mathrel{\prec_L}}
\newcommand{\geqLst}{\mathrel{\succeq_{L^*}}}
\newcommand{\gLst}{\mathrel{\succ_{L^*}}}
\newcommand{\leqLst}{\mathrel{\preceq_{L^*}}}
\newcommand{\lLst}{\mathrel{\prec_{L^*}}}

%\newcounter{lecture}
%\newcommand{\lecturefl}[1]{   % use with foiltex landscape
%% \addtocounter{lecture}{1}
% \refstepcounter{lecture}
% \setcounter{equation}{0}
% \setcounter{page}{1}
% \renewcommand{\theequation}{\arabic{equation}}
% \renewcommand{\thepage}{\arabic{lecture}--\arabic{page}}
% \raggedright
% \parindent 0pt
% \rightfooter{\thepage}
% \leftheader{}
% \rightheader{}
% \LogoOff
% \input header 
% \begin{center}
%% {\Large \bfseries Lecture \arabic{lecture} \\*[\bigskipamount] {#1}}
%{\Large \bfseries \arabic{lecture}.  {#1}}
% \end{center}
% \MyLogo{#1}
%}

%\newcommand{\lectureflstar}[1]{   % use with foiltex landscape
% \setcounter{equation}{0}
% \setcounter{page}{1}
% \renewcommand{\theequation}{\arabic{equation}}
% \renewcommand{\thepage}{\arabic{page}}
% \raggedright
% \parindent 0pt
% \rightfooter{\thepage}
% \leftheader{}
% \rightheader{}
% \LogoOff
% \input header 
% \begin{center}
% {\Large \bfseries #1}
% \end{center}
% \MyLogo{#1}
%}
%\newcounter{oursection}
%\newcommand{\frametitle}[1]{  % for use with foiltex landscape
% \addtocounter{oursection}{1}
%% \setcounter{equation}{0}
% \foilhead[-1.0cm]{#1}
% \LogoOn
%}

\newenvironment{algdesc}%
   {\begin{list}{}{%
   \setlength{\rightmargin}{0\linewidth}%
   \setlength{\leftmargin}{.05\linewidth}}%
   \sffamily\small
   \item[]{\setlength{\parskip}{0ex}\hrulefill\par%
   \nopagebreak{}}}%
   {{\setlength{\parskip}{-1ex}\nopagebreak\par\hrulefill} \end{list}}

\newenvironment{colm}{\left[\begin{array}{c}}{\end{array}\right]}
\newenvironment{colv}{\left(\begin{array}{c}}{\end{array}\right)}


\newcommand{\oh}{\frac12}
\newcommand{\st}{\text{subject to}}
\newcommand{\gfb}{\nabla f(\bar x)}
\newcommand{\hfb}{H(\bar x)}

\newtheorem{theorem}{Theorem}[section]
\newtheorem{remark}[theorem]{Remark}%[section]
\newtheorem{definition}[theorem]{Definition}%[section]
\newtheorem{proposition}[theorem]{Proposition}%[section]
\newtheorem{lemma}[theorem]{Lemma}%[section]
\newtheorem{corollary}[theorem]{Corollary}%[section]
\newtheorem{assumption}{Assumption}
\newtheorem{claim}{Claim}
\newtheorem{exam}{Example}
\newenvironment{solution}
  {\renewcommand\qedsymbol{$\square$}\begin{proof}[\textbf{Solution}]}
  {\end{proof}}
\renewcommand{\proofname}{\textbf{Proof}}

% Colors
\newcommand{\red}[1]{\textcolor{red}{#1}}
\newcommand{\blue}[1]{\textcolor{blue}{#1}}
\newcommand{\green}[1]{\textcolor{green}{#1}}

% Handy math notations
\newcommand{\grad}{\nabla}
\newcommand{\hess}{\nabla^2}
\newcommand{\tr}{\text{tr}}
\newcommand{\pd}[2][]{ \frac{\partial #1}{\partial #2}} % Partial derivatives
\renewcommand{\d}{{\rm d}}
\newcommand{\ddt}{\frac{\d}{\d t}}
\newcommand{\half}{\frac{1}{2}} % 1/2
\newcommand{\inv}{^{-1}}        % inverse
\newcommand{\T}{^\top}          % transpose

\begin{document}
\title{IOE 611: Final Exam}
\author{Yulun Zhuang}
\maketitle
%**********************************
\section*{Problem 1}
(a)
Let $B(0, 1) = \{x\in\real^n \mid \|x\|^2_2\leq 1\}$, we have
\begin{align*}
  E &= \{x\in\real^n \mid\|Q^{-\half}x\|^2_2 = x\T Q^{-1}x \leq 1\}\\
  &= Q^{\half} B(0, 1)\\
  &= \{Q^{\half} x \mid \|x\|^2_2\leq 1\}
\end{align*}

Hence, $E\subseteq S$ is equivalent to $B(0, 1) \subseteq Q^{-\half}S = \{Q^{-\half}x \mid |a\T_ix|\leq 1, i=1, \dots, k\}$.

Note that $Q^{-\frac{1}{2}} S=\{y \in \mathbb{R}^n:|a_i^{\top} Q^{\frac{1}{2}} y|^2 \leq 1, i=1, \ldots, k\}$

If $B(0,1) \subseteq Q^{\frac{1}{2}} S$, let $y_i=\frac{Q^{\frac{1}{2}} a_i}{\|Q^{\frac{1}{2}} a_i\|_2}$, then we have
$$
\begin{aligned}
& a_i^{\top} Q^{\frac{1}{2}} y_i=a_i^{\top} Q a_i / \sqrt{a_i^{\top} Q a_i}=\left(a_i^{\top} Q a_i\right)^{\frac{1}{2}} \leq 1, \quad \forall 1 \leq i \leq k .
\end{aligned}
$$

If $a_i^{\top} Q a_i \leq 1, \forall 1 \leq i \leq k$, by Cauchy-Schwarz inequality, for any $y \in B(0,1), 1 \leq i \leq k$
$$
\begin{aligned}
a_i^{\top} Q^{\frac{1}{2}} y & =\left\langle Q^{\frac{1}{2}} a_i, y\right\rangle \\
& \leq\left\|Q^{\frac{1}{2}} a_i\right\|_2 \cdot\|y\|_2 \\
& =\sqrt{a_i^{\top} Q a_i} \cdot\|y\|_2 \\
& \leq 1 .
\end{aligned}
$$

Therefore, $B(0,1) \leq Q^{-\frac{1}{2}} S$.

(b)
Recall that the volume of $E = C_n\sqrt{\det Q}$, where $C_n > 0$ is a constant w.r.t. $Q$. Since $y=\log x$ is increasing in $x>0$, maximizing the volume of $E$ given $E\subseteq S$ is equivalent to
\begin{align*}
  \max\quad & \log\det Q\\
  \text{s.t.}\quad & a_i\T Q a_i \leq 1, \forall 1\leq i \leq k.
\end{align*}

The Lagrange is
$$
\begin{aligned}
L(Q, \lambda) & =\log \operatorname{det} Q+\sum_{i=1}^k \lambda_i(1-a_i^{\top} Q a_i) \\
& =\log \operatorname{det} Q+\sum_{i=1}^k \lambda_i-\operatorname{tr}((\sum_{i=1}^k \lambda_i a_i a_i^{\top}) Q),
\end{aligned}
$$


Therefore, since $\frac{\partial}{\partial Q} \log \operatorname{det} Q=Q^{-\top}=Q^{-1}, \frac{\partial}{\partial Q} \operatorname{tr}\left(A^{\top} Q\right)=A$,
$$
\frac{\partial}{\partial Q} L(Q, \lambda)=Q^{-1}-\sum_{i=1}^k \lambda_i a_i a_i^{\top},
$$


The dual problem is:
\begin{align*}
  \min_{\lambda\in\real^n} \quad & -\log \operatorname{det}(\sum_{i=1}^k \lambda_i a_i a_i^{\top})+\sum_{i=1}^k \lambda_i-n\\
  \text{s.t.}\quad & \lambda \geq 0.
\end{align*}



\clearpage
\section*{Problem 2}

Let $y=0$, we have $Ay = 0, y\T \hess f(\bar{x})y = 1 < 1$, so the Slater's condition is satisfied.

The Lagrangian is
\[
  L(y, \lambda, \nu) = \grad f(\bar{x})\T y - \lambda(1 - y\T \hess f(\bar{x})y) + \nu\T Ay,
\]

The KKT conditions at the optimal solution $\bar{y}$ are
\begin{align}
  & \grad f(\bar{x}) + 2\lambda\hess f(\bar{x})\bar{y} + A\T \nu = 0\label{eq:lag-grad}\\
  & \bar{y}\T\hess f(\bar{x})\bar{y}\leq 1\label{eq:primal-ineq}\\
  & A\bar{y} = 0\\
  & \lambda \geq 0\label{eq:dual-ineq}\\
  & \lambda (\bar{y}\T\hess f(\bar{x})\bar{y} - 1) = 0\label{eq:comp-slack}
\end{align}

From \eqref{eq:comp-slack} we know that
\begin{align*}
  &\grad f(\bar{x}) + \hess f(\bar{x}) \Delta x + A\T w = 0\\
  &A\T \Delta x = 0
\end{align*}

Let $\bar{y} = \frac{\Delta x}{2\lambda}, w=\nu$, we have $A\bar{y} = 0$ and \eqref{eq:lag-grad} is satisfied.

Let $\lambda = \half \sqrt{\Delta x\T \hess f(\bar{x})\Delta x} = \half \lambda (\bar{x})$, we have \eqref{eq:dual-ineq}, \eqref{eq:primal-ineq} and \eqref{eq:comp-slack} are satisfied, given that
$$\bar{y}\T\hess f(x)y = \frac{\Delta x\T \hess f(\bar{x})\Delta x}{\Delta x\T \hess f(\bar{x})\Delta x} = 1$$

Therefore, $\bar{y} = \frac{\Delta x}{\lambda(\bar{x})}$.


\clearpage
\section*{Problem 3}

(a)
Since (P1) is convex, $f_i(x)$ is convex for all $0 \leq i \leq m$.

Since the maximum of convex functions is convex, we have
\begin{align*}
  &f_i^{+}(x) =\max \{0, f_i(x)\}\\
  \Rightarrow\quad &\max _{1 \leq i \leq m} f_i^{+}(x) \text{ is convex.}
\end{align*}

Since $y=|x|$ is convex and $a_j^{\top} x-b_j$ is affine, we have
$$
\begin{aligned}
& \left|a_j^{\top} x-b_j\right| \text { is convex} \\
\Rightarrow \quad & \max _{1\leq j\leq p}\left|a_j^{\top} x-b_j\right| \text { is convex. }
\end{aligned}
$$

Given $t_1, t_2>0$, the sum of convex functions is also convex.

Thus, $f(x):=f_0(x)+t_1 \max_{1 \leq i \leq m} f_1^{+}(x)+t_2 \max _{1 \leq j \leq p} |a_j^{\top} x-t_j|$ is convex.

(b)

$"\Rightarrow"$

Let $x^*$ be the optimal solution of (P1-relaxed), we can define
$$
\begin{aligned}
& y^*=\max _{1 \leq i \leq m} f^{+}\left(x^*\right), \ z^*=\max _{1 \leq j \leq p}\left|a_j^{\top} x^*-b_j\right|,
\end{aligned}
$$
such that $\left(x^*, y^*, z^*\right)$ is a feasible solution of (P2).

First we show the objectives are equivalent at $\left(x^*, y^*, z^*\right)$
$$v_2\left(x^*, y^*, z^*\right) = f_0\left(x^*\right) + t_1 y^* + t_2 z^* = f\left(x^*\right) = v_1\left(x^*\right).$$

Note that for any feasible $(x, y, z)$ of (P2),
$$
\begin{aligned}
& y \geq \max \left\{0, f_i(x), \dots, f_m(x)\right\}=\max _{1 \leq i \leq m} f_i^{+}(x), \\
& z \geq \max _{1 \leq j \leq p}\left|a_j^{\top} x-b_j\right|\\
\Rightarrow \quad & v_2(x, y, z) \geq v_1(x).
\end{aligned}
$$


Since $x^*$ is the minima of (P1-relaxed) $\forall x$, we have $v_1(x) \geq v_1\left(x^*\right)$.

For any feasible $(x, y, z)$ of (P2), $v_2(x, y, z) \geq v_1(x) \geq v_1\left(x^*\right)=v_2\left(x^*, y^*, z^*\right)$.

Thus, $\left(x^*, y^*, z^*\right)$ is the optimal of (P2).

$"\Leftarrow"$

If $\left(x^*, y^*, z^*\right)$ is the optimal of (P2), we must have
$$y^*=\max _{1 \leq i \leq m} f_i^{+}\left(x^*\right), \quad z^*=\max _{1 \leq j \leq p}\left|a_j^{\top} x^*-b_j\right|.$$

Otherwise, we can find
$$\bar{y}=\max _{1 \leq i \leq m} f_i^{+}\left(x^*\right),\quad \bar{z}=\max _{1 \leq j \leq p}\left|a_j^{\top} x^*-b_j\right|,$$ 
such that $\left(x^*, \bar{y}, \bar{z}\right)$ is feasible in
(P2), and $y^*-\bar{y} \geq 0, z^*-\bar{z} \geq 0$.

However, the objective value is $v_2\left(x^*, \bar{y}, \bar{z}\right)<v_2\left(x^*, y^*, z^*\right)$, which leads to the contradiction.

For any $x$,
$$
\begin{aligned}
v_1(x) 
& =v_2(x, \max _{1 \leq i \leq m} f_i^{+}(x), \max _{1 \leq j \leq p}\left|a_j^{\top} x-b_j\right|) \\
& \geq v_2\left(x^*, y^*, z^*\right)\\
& =v_1\left(x^*\right)
\end{aligned}
$$

Thus, $x^*$ is the optimal solution of (P1-relaxed) and the objective value $v_1^*=v_2^*$.

Finally we can conclude that (P1-relaxed) is equivalent to (P2).

(c)
The Lagrangian of (P2) is
\begin{align*}
  L\left(x, y, z, \lambda^{(1)}, \lambda^{(2)}, \lambda^{(3)}\right)= & f_0(x)+t_1 y+t_2 z+\sum_{i=1}^m \lambda_i^{(1)}\left(f_i(x)-y\right) \\
  & +\sum_{j=1}^m \lambda_j^{(2)}\left(a_j^{\top} x-b_j-z\right)+\sum_{j=1}^m \lambda_{j+p}^{(2)}\left(b_j-a_j^{\top} x-z\right) \\
  & -\lambda_{1}^{(3)} y - \lambda_{2}^{(3)} z,
\end{align*}

The partial derivatives are
\begin{align*}
  &\frac{\partial L}{\partial x}=\nabla f_0(x)+\sum_{i=1}^m \lambda_i^{(m)} \nabla f_i(x)+\sum_{j=1}^m\left(\lambda_j^{(2)}-\lambda_{j+m}^{(2)}\right) a_j\\
  & \frac{\partial L}{\partial y}=t_1-\sum_{i=1}^m \lambda_i^{(1)}-\lambda_{1}^{(3)} \\
  & \frac{\partial L}{\partial z}=t_2-\sum_{j=1}^{2 p} \lambda_j^{(2)}-\lambda_2^{(3)}
\end{align*}

The dual problem of (P2) can be formulated as (D2)
\begin{alignat*}{3}
  &\max \quad &\bar{g}(\lambda^{(1)}, \lambda^{(2)}, \lambda^{(3)}) = & -(f_0+\sum_{i=1}^m \lambda_i^{(1)} f_i)^{*}(-\sum_{j=1}^m(\lambda_j^{(2)}-\lambda_{j+p}^{(2)}) a_j) +\sum_{j=1}^m(\lambda_{j+p}^{(2)}-\lambda_j^{(2)}) b_j\\
  & \text { s.t.} \quad 
  & t_1=&\sum_{i=1}^m \lambda_i^{(1)}+\lambda_1^{(3)}, \\
  && t_2=&\sum_{j=1}^{2 p} \lambda_j^{(2)}+\lambda_2^{(3)}, \\
  && \lambda^{(1)}, &\lambda^{(2)}, \lambda^{(3)} \geq 0
\end{alignat*}

For any $x$, we can let $y, z$ to be sufficiently large, then the inequality constraints are strict, and the Slater's condition will be satisfied.
Therefore, (P2) and (D2) have strong duality.

Note that for the optimal solution $x^*$ in (P1), $(x^*, 0, 0)$ is feasible in (P2) and $f_0(x^*) = v_2(x^*, 0, 0)$. By the strong duality,
\[
  d_1^* = f_0(x^*) = v_2(x^*, 0, 0) \geq v_2^* = d_2^*
\]

(d)
The KKT conditions of (P1) are
\begin{align}
  & \nabla f_0\left(x^*\right)+\sum_{i=1}^m \lambda_i^* \nabla f_i\left(x^*\right)+\sum_{j=1}^p \nu_j^* a_j=0\label{eq:lag-grad-p3}\\
  & f_i\left(x^*\right) \leq 0, \  1 \leq i \leq m \label{eq:primal-ineq-p3}\\
  & a_j^{\top} x-b_j=0, \  1 \leq j \leq p \label{eq:primal-eq-p3}\\
  & \lambda_i^* f_i\left(x^*\right)=0, \  1 \leq i \leq m \label{eq:comp-slack-p3}\\
  & \lambda^* \geq 0 \label{eq:dual-ineq-p3}
\end{align}

If $(x^*, \lambda^*, \nu^*)$ is the primal-dual pair of (P1) with $\sum_{i=1}^{m} < t_1, \sum_{j=1}^{n} |\nu_j^*| < t_2$, we can define the followings to satisfy the KKT conditions of (P2)
\begin{align*}
  & \bar{x} = x^*, \bar y = 0, \bar z=0\\
  & \bar{\lambda}_i^{(1)} = \lambda_i^*,\  1\leq i\leq m\\
  & \bar{\lambda}_j^{(2)} = (\nu_j^*)^+,\  \bar{\lambda}_{j+p}^{(2)} = (\nu_j^*)^-, \ 1\leq j\leq p\\
  & \bar{\lambda}_1^{(3)} = t_1 - \sum_{i=1}^{m}\lambda_i^*>0\\
  & \bar{\lambda}_2^{(3)} = t_2 - \sum_{j=1}^{p} |\nu_j^*| >0,
\end{align*}

such that $(x^*, 0, 0, \bar{\lambda}^{(1)}, \bar{\lambda}^{(2)}, \bar{\lambda}^{(3)})$ forms a primal-dual pair of (P2) and (D2), and $(\bar{\lambda}^{(1)}, \bar{\lambda}^{(2)}, \bar{\lambda}^{(3)})$ is the optimal solution of (D2). 

Note that $\bar{\lambda}^{(3)} > 0$, if $(\bar x, \bar{y}, \bar{z})$ is the optimal solution of (P2), we must have $\bar{\lambda}_1^{(3)} \bar{y} = 0, \bar{\lambda}_2^{(3)} \bar{z} =0\  \Rightarrow\ \bar{y} = \bar{z} = 0$, indicating that all optimal solution of (P2) must satisfy 
\begin{align}
  y = z = 0. \tag{$*$}
\end{align}

From part (b), if $\bar{x}$ is the optimal solution of (P1-relax), then $(\bar{x}, \bar{y}, \bar{z})$ is the optimal solution to (P2).

However, by conclusion ($*$), we know that every solution to (P2) must have the form $(x, 0,0)$, so we have
$$
\left\{\begin{array}{l}
0=\bar{y}=\max _i f_i^{+}(\bar{x})\ \Leftrightarrow\  \forall 1 \leq i \leq m,\  f_i(\bar{x})=0, \\
0=\bar{z}=\max _j\left|a_j^{\top} \bar{x}-b_j\right|\  \Leftrightarrow\  \forall 1 \leq j \leq p, \  a_j^{\top} \bar{x}-b_j=0,
\end{array}\right.
$$
which shows that $\bar{x}$ is feasible in (P1), and the objective value $f(\bar{x}) = f_0(\bar{x})$.

However, for any feasible solution $x$ of (P1), $x$ is also feasible in (P1-relax), and $f(x) = f_0(x)$, so the optimal value of (P1-relax) $v_1^*$ is smaller them that of (P1) $v_0^*$.

Consequently, $f_0(\bar{x})=v_1^* \leq v_0^*$, but $\bar{x}$ is feasible in (P1) $\Rightarrow\ f_0(\bar{x}) \geq v_0^*$, so $f_0(\bar{x})=v_0^*$, showing that $\bar{x}$ is the optimal solution of (P1).


\clearpage
\section*{Problem 4}

(a)
Given the positive definite and tridiagonal matrix $M\in\symm^n$,
\begin{align*}
  M = 
  \begin{bmatrix}
    a_1 & b_1 &     &   & \\
    b_1 & a_2 & b_2 &   & \\
        & b_2 & a_3 &\ddots&\\
        &     &\ddots&\ddots& b_{n-1}\\
        &     &     & b_{n-1} & a_n
  \end{bmatrix}
  = LL\T, \ 
  L = 
  \begin{bmatrix}
    u_1 &     &     & \\
    v_1 & u_2 &     & \\
        & v_2 & u_3 & \\
        &     & \ddots & \ddots \\
        &     & & v_{n-1} & u_n
  \end{bmatrix}
\end{align*}

and by the Cholesky factorization, we have
\begin{align*}
  & u_1^2 = a_1\\
  & u_i v_i = b_i, \ 1\leq i\leq n-1\\
  & v_i^2 + u_{i+1}^2 = a_{i+1}, \ 1 \leq i\leq n-1
\end{align*}

Then we can calculate $u, v$ in the following order
\begin{itemize}
  \item $u_1 = \sqrt{a_1}$
  \item $v_1 = \frac{b_1}{u_1}, u_2 = \sqrt{a_2 - v_1^2}$
  \item $v_2 = \frac{b_2}{u_2}, u_3 = \sqrt{a_3 - v_2^2}$
  \item \dots
  \item $v_{n-1} = \frac{b_{n-1}}{u_{n-1}}, u_n = \sqrt{a_n - v_{n-1}^2}$
\end{itemize}

Thus, the time complexity becomes $O(n)$.

(b)
Given that $\alpha, \beta > 0, y=x^2$ is convex, and $a_i\T x - b_i \text{ for }i=1, \dots, k, x_{j+1} - x_j \text{ for } j = 1, \dots, n-1$ are affine, we know the objective function $f(x)$ is convex.

Since $f(x)$ is differentiable, $x^*$ is optimal if and only if $\grad f(x^*)=0$.
Note that
$$
\begin{aligned}
& \frac{\partial}{\partial x_1} \sum_{j=1}^{n-1}\left(x_j-x_{j+1}\right)^2=2\left(x_l-x_2\right) \\
& \frac{\partial}{\partial x_l} \sum_{j=1}^{n-1}\left(x_j-x_{j+1}\right)^2=2\left(2 x_l-x_{l-1}-x_{l+1}\right),\ 2 \leq l \leq n-1 \\
& \frac{\partial}{\partial x_n} \sum_{j=1}^{n-1}\left(x_j-x_{j+1}\right)^2=2\left(x_n-x_{n-1}\right),
\end{aligned}
$$

We can pick $C$ to be
\begin{align*}
  C =
  \begin{bmatrix}
     1 & -1 &    & & & \\
    -1 &  2 & -1 & & & \\
       & -1 & 2 & \ddots &&\\
       &    & \ddots &\ddots & -1\\
       &    &   & -1 & 2
  \end{bmatrix}
\end{align*}
such that $\grad \sum_{j=1}^{n-1}(x_j - x_{j+1})^2 = 2 C x$.
Therefore,
$$
\begin{aligned}
\nabla f(x) 
& =\sum_{i=1}^k a_i(a_i^{\top} x-b_i)+2 \alpha x+2 \beta C x \\
& =(\sum_{i=1}^k a_i a_i^{\top}+2 \alpha I+2 \beta C) x-\sum_{i=1}^k a_i b_i \\
\Rightarrow \quad x^* 
& =(\sum_{i=1}^k a_i a_i^{\top}+2 \alpha I+2 \beta C)^{-1}(\sum_{i=1}^k a_i b_i)
\end{aligned}
$$

(c)
We can now calculate $x^*$ in the following order
\begin{itemize}
  \item Apply Cholesky factorization to $2\alpha I + 2\beta C = LL\T$, since the LHS is tridiagonal $\Leftarrow O(n)$;
  \item Calculate $B\inv = (LL\T)\inv$. Since $L$ is in lower tridiagonal form, let $L\inv$ be
  \begin{align*}
    L\inv = 
    \begin{bmatrix}
      s_1 &     &     & \\
      t_1 & s_2 &     & \\
          & t_2 & s_3 & \\
          &     & \ddots & \ddots \\
          &     & & t_{n-1} & s_n
    \end{bmatrix}
    \Rightarrow 
    \begin{cases}
      s_i = u_i\inv & 1\leq i \leq n\\
      t_i = -\frac{v_i s_i}{u_{i+1}} & 1\leq i\leq n-1
    \end{cases}
  \end{align*}

  Thus, the cost of inverting $L$ is $O(n)$. 

  Since $B\inv$ should also be tridiagonal, the cost of inverting it is also $O(n)$.
  \item Calculate $\bar{b} = \sum_{i=1}^{k} a_i b_i \Leftarrow O(kn)$
  \item Let $A = [a_1\ \cdots \ a_k] \in\real^{n\times k}$, we can write $x^*$ into the following form
  \begin{align*}
    (B + AA\T) x^* = \bar{b} \quad \Leftrightarrow \quad
    \begin{bmatrix}
      B & A\\ A\T & -I
    \end{bmatrix}
    \begin{bmatrix}
      x^*\\ y^*
    \end{bmatrix}
    = 
    \begin{bmatrix}
      \bar{b}\\ 0
    \end{bmatrix},
  \end{align*}
  
  We can solve the linear system via block elimination
  \begin{align*}
    \begin{cases}
      (I + A\T B\inv A) y^* = A\T B\inv \bar{b}\\      
      x^* = B\inv \bar{b} - B\inv A y^* = L\inv L^{-\top} \bar{b} - L\inv L^{-\top} Ay^*
    \end{cases}
  \end{align*}
\end{itemize}
  
Therefore, the total time complexity is $O(k^2n + k^3)$.

\clearpage
\section*{Problem 5}

(a) 
% To solve the centering problem using Newton's method, we first compute the gradient and Hessian of the objective function. 
The gradient and Hessian of \(f(x)\) are
\begin{align*}
  \nabla f(x) &= c - \frac{1}{x}\\
  \nabla^2 f(x) &= \text{diag}\left(\frac{1}{x^2}\right),
\end{align*}
where \(\frac{1}{x}\) and \(\frac{1}{x^2}\) are element-wise, i.e., \(\left[\frac{1}{x}\right]_j = \frac{1}{x_j}\), \(\left[\frac{1}{x^2}\right]_j = \frac{1}{x_j^2}\).
The equality constraint Newton system is derived from the Karush-Kuhn-Tucker (KKT) conditions
\[
\begin{aligned}
    \nabla f(x) + A^\top \lambda &= 0 \\
    Ax - b &= 0,
\end{aligned}
\]
where \(\lambda \in \mathbb{R}^m\) are the Lagrange multipliers.
Rewriting in matrix form
\[
\begin{bmatrix}
    \nabla^2 f(x) & A^\top \\
    A & 0
\end{bmatrix}
\begin{bmatrix}
    \Delta x \\
    \Delta \lambda
\end{bmatrix}
=
\begin{bmatrix}
    -\nabla f(x) \\
    b - Ax
\end{bmatrix},
\]
where \(\Delta x\) is the Newton step, and \(\Delta \lambda\) is the Newton decrement.

To solve the KKT system efficiently, we exploit the structure of the equations. The Schur complement can be used to eliminate \(\Delta \lambda\) and solve for \(\Delta x\) directly
\[
\Delta x = -H^{-1}(\nabla f(x) + A^\top \Delta \lambda),
\]
where \(H = \nabla^2 f(x)\).
Substituting back, we solve
\[
(AH^{-1}A^\top)\Delta \lambda = A H^{-1} \nabla f(x) - (b - Ax).
\]

This approach avoids directly inverting the full KKT matrix, improving the computational efficiency.
At each iteration:
\begin{itemize}
  \item Solve the Newton system to obtain \(\Delta x\) and \(\Delta \lambda\).
  \item Perform a backtracking line search to ensure feasibility, reducing the step size \(\alpha\) until \(x + \alpha \Delta x > 0\).
  \item Update the primal and dual variables:
  \(
  x \gets x + \alpha \Delta x, \  \lambda \gets \lambda + \alpha \Delta \lambda.
  \)
\end{itemize}

The stopping criterion is:
\(
{\|\Delta x\|^2}/2 \leq \epsilon,
\)
where \(\epsilon\) is a small tolerance.

As shown in Fig.~\ref{fig:centering}, the implemented method converges quadratically on the generated instance.


\begin{figure}[htb]
  \includegraphics[width=0.7\columnwidth]{"final_p5_a"}
  \centering
  \caption{Convergence of Newton Method}
  \label{fig:centering}
\end{figure}


(b)
As shown in Fig.~\ref{fig:barrier}, smaller \(\mu\) leads to more Newton steps and thus slower convergence, while larger \(\mu\) leads to fewer Newton steps but higher duality gap. Based on this observation, we pick \(\mu = 200\) in the implementation.

\begin{figure}[htb]
  \includegraphics[width=0.7\columnwidth]{"final_p5_b"}
  \centering
  \caption{Convergence of Barrier Method}
  \label{fig:barrier}
\end{figure}

\end{document}