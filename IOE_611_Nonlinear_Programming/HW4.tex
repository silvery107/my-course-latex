\documentclass[11pt]{article}
\usepackage{geometry}                
\geometry{letterpaper,tmargin=1in,bmargin=1in,lmargin=1in,rmargin=1in} 
\usepackage[parfill]{parskip}    % Activate to begin paragraphs with an empty line rather than an indent
\usepackage{graphicx}
\usepackage{amsmath, amsfonts, amsthm, amssymb} 
\usepackage[shortlabels]{enumitem}
\usepackage{xcolor}
\usepackage{mathtools}

\parskip = 0.1in

\pagestyle{myheadings}
\markright{Homework of IOE 611 Nonlinear Programming\hfill Yulun Zhuang \hfill}

\input{611defs}

\newcommand{\oh}{\frac12}
\newcommand{\st}{\text{subject to}}
\newcommand{\gfb}{\nabla f(\bar x)}
\newcommand{\hfb}{H(\bar x)}

\newtheorem{theorem}{Theorem}[section]
\newtheorem{remark}[theorem]{Remark}%[section]
\newtheorem{definition}[theorem]{Definition}%[section]
\newtheorem{proposition}[theorem]{Proposition}%[section]
\newtheorem{lemma}[theorem]{Lemma}%[section]
\newtheorem{corollary}[theorem]{Corollary}%[section]
\newtheorem{assumption}{Assumption}
\newtheorem{claim}{Claim}
\newtheorem{exam}{Example}
\newenvironment{solution}
  {\renewcommand\qedsymbol{$\square$}\begin{proof}[\textbf{Solution}]}
  {\end{proof}}
\renewcommand{\proofname}{\textbf{Proof}}

\newcommand{\red}[1]{\textcolor{red}{#1}}
\newcommand{\blue}[1]{\textcolor{blue}{#1}}
\newcommand{\green}[1]{\textcolor{green}{#1}}
\newcommand{\grad}{\nabla}
\newcommand{\hess}{\nabla^2}
\newcommand{\tr}{\text{tr}}

\newcommand{\dd}{\mathrm{d}}
\newcommand{\RR}{\mathbb{R}}
\newcommand{\NN}{\mathbb{N}}
\newcommand{\ZZ}{\mathbb{Z}}
\newcommand{\bS}{\mathbb{S}}

\newcommand{\pd}[2][]{ \frac{\partial #1}{\partial #2}} % Partial derivatives
\renewcommand{\d}{{\rm d}}
\newcommand{\ddt}{\frac{\d}{\d t}}
\newcommand{\half}{\frac{1}{2}}
\newcommand{\inv}{^{-1}}
\newcommand{\T}{^\top}

\begin{document}
\title{IOE 611: Homework 4}
\author{Yulun Zhuang}
\maketitle
%**********************************
\section*{Problem 1}

\clearpage
\section*{Problem 2}

\clearpage
\section*{Problem 3}

\clearpage
\section*{Problem 4}
\textit{Weak duality for unbounded and infeasible problems}. 
The weak duality inequality, $d^* \leq p^*$,
clearly holds when $d^* = -\infty$ or $p^* = \infty$. Show that it holds in the other two cases as well: If $p^* = -\infty$, then we must have $d^* = -\infty$, and also, if $d^* = \infty$, then we must have $p^* = \infty$.

\clearpage
\section*{Problem 5}
\textit{Suboptimality of a simple covering ellipsoid}. 
Recall the problem of determining the minimum
volume ellipsoid, centered at the origin, that contains the points $a_1, \dots, a_m\in\real^n$
\begin{align*}
  \text{minimize}\quad& f_0(X) = \log\det(X^{-1})\\
  \text{subject to}\quad & a_i\T X a_i \leq 1, \ i=1,\dots, m,
\end{align*}
with $\dom f_0 = \symm^n_{++}$.
We assume that the vectors $a_1, \dots, m \text{ span } \real^n$ (which implies that the problem is bounded below).

(a) Show that the matrix
\[
X_{sim} = \left(\sum_{k=1}^{m}a_ka_k\T\right)^{-1},
\]
is feasible.

(b) Now we establish a bound on how suboptimal the feasible point $X_{sim}$ is, via the dual
problem,
\begin{align*}
  \text{minimize}\quad& \log\det(\sum_{i=1}^{m} \lambda_i a_i a_i\T) - \ones\T \lambda + n\\
  \text{subject to}\quad & \lambda \succeq 0,
\end{align*}
with the implicit constraint $\sum_{i=1}^{m} \lambda_i a_i a_i\T \succ 0$.
To derive a bound, we restrict our attention to dual variables of the form $\lambda = t\ones$, where $t > 0$. Find (analytically) the optimal value of $t$, and evaluate the dual objective at this $\lambda$. Use this to prove that the volume of the ellipsoid $\{u\mid u\T X_{sim}u\leq 1\}$ is no more than a factor $(m/n)^{n/2}$ more than the volume of the minimum volume ellipsoid.

\end{document}
