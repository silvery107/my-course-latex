\documentclass[11pt]{article}
\usepackage{geometry}                
\geometry{letterpaper,tmargin=1in,bmargin=1in,lmargin=1in,rmargin=1in} 
\usepackage[parfill]{parskip}    % Activate to begin paragraphs with an empty line rather than an indent
\usepackage{graphicx}
\usepackage{amsmath, amsfonts, amsthm, amssymb} 
\usepackage[shortlabels]{enumitem}
\usepackage{xcolor}
\usepackage{mathtools}

\parskip = 0.1in

\pagestyle{myheadings}
\markright{Homework of IOE 611 Nonlinear Programming\hfill Yulun Zhuang \hfill}

 % some traditional definitions that can be blamed on craig barratt
 \newcommand{\BEAS}{\begin{eqnarray*}}
\newcommand{\EEAS}{\end{eqnarray*}}
\newcommand{\BEA}{\begin{eqnarray}}
\newcommand{\EEA}{\end{eqnarray}}
\newcommand{\BEQ}{\begin{equation}}
\newcommand{\EEQ}{\end{equation}}
\newcommand{\BIT}{\begin{itemize}}
\newcommand{\EIT}{\end{itemize}}

% text abbrevs
\newcommand{\eg}{e.g.}
\newcommand{\ie}{i.e.}

% std math stuff
\newcommand{\ones}{\mathbf 1}
\newcommand{\real}{{{\mathbb{R}}}}
\newcommand{\integer}{{{\mathbb{Z}}}}
\newcommand{\complex}{{{\mathbb{C}}}}
\newcommand{\symm}{{{\mathbb{S}}}}  % symmetric matrices
%
% lin alg stuff
\newcommand{\Span}{\mbox{\textrm{span}}}
\newcommand{\range}{{\mathcal R}}
\newcommand{\nullspace}{{\mathcal N}}
\newcommand{\Rank}{\mathop{\textbf{rank}}}
\newcommand{\Tr}{\mathop{\textbf{tr}}}
\newcommand{\cond}{\mathop{\textbf{cond}}}
\newcommand{\diag}{\mathop{\textbf{diag}}}
\newcommand{\lambdamax}{\lambda_{\rm max}}
\newcommand{\lambdamin}{\lambda_{\rm min}}

% probability stuff
\newcommand{\Prob}{\mathop{\textbf{prob}}}
\newcommand{\Expect}{\mathop{\textbf{E{}}}}
\newcommand{\var}{\mathop{\textbf{var}}} % variance
% not sure why we have \Expect and \Prob but \var ???

% convexity & optimization stuff
\newcommand{\Co}{\mathop {\textbf{conv}}} % convex hull
\newcommand{\argmin}{\mathop{\rm argmin}}
\newcommand{\argmax}{\mathop{\rm argmax}}
\newcommand{\epi}{\mathop{\textbf{epi}}}
%\newcommand{\hypo}{\mathop{\textbf{hypo}}}}

% sup and inf that look OK in saddle-point form!
%\newcommand{\ourinf}{\mathop{\raisebox{0ex}[0ex][.4ex]{\,inf\,}}}
%\newcommand{\oursup}{\mathop{\raisebox{0ex}[0ex][.4ex]{\,sup\,}}}
\newcommand{\ourinf}{\mathop{\,\mathrm{inf}\, {\rule[-.5ex]{0ex}{0ex}}}}
\newcommand{\oursup}{\mathop{\,\mathrm{sup}\, {\rule[-.5ex]{0ex}{0ex}}}}
%makes latex believe that inf and sup both extend .4ex below
%the baseline

\newcommand{\dist}{\mathop{\textbf{dist}}}
\newcommand{\vol}{\mathop{\textbf{vol}}} % volume
\newcommand{\Card}{\mathop{\textbf{card}}} % cardinality
\newcommand{\sign}{\mathop{\textbf{sign}}}

\newcommand{\dom}{\mathop{\textbf{dom}}} % domain
\newcommand{\aff}{\mathop{\textbf{aff}}} % affine hull
\newcommand{\cl}{\mathop{\textbf{cl}}} % closure
\newcommand{\intr}{\mathop{\textbf{int}}} % interior
\newcommand{\relint}{\mathop{\textbf{rel int}}} % relative interior
\newcommand{\bd}{\mathop{\textbf{bd}}} % boundary

%why do we have the following but not \nust?
\newcommand{\xst}{x^\star}
\newcommand{\lambdast}{\lambda^\star}

% defs for cones & generalized inequalities
% these seem kind of awkward; should fix some day
% rewrite them to use args?
\newcommand{\geqK}{\mathrel{\succeq_K}}
\newcommand{\gK}{\mathrel{\succ_K}}
\newcommand{\leqK}{\mathrel{\preceq_K}}
\newcommand{\lK}{\mathrel{\prec_K}}
\newcommand{\geqKst}{\mathrel{\succeq_{K^*}}}
\newcommand{\gKst}{\mathrel{\succ_{K^*}}}
\newcommand{\leqKst}{\mathrel{\preceq_{K^*}}}
\newcommand{\lKst}{\mathrel{\prec_{K^*}}}
\newcommand{\geqL}{\mathrel{\succeq_L}}
\newcommand{\gL}{\mathrel{\succ_L}}
\newcommand{\leqL}{\mathrel{\preceq_L}}
\newcommand{\lL}{\mathrel{\prec_L}}
\newcommand{\geqLst}{\mathrel{\succeq_{L^*}}}
\newcommand{\gLst}{\mathrel{\succ_{L^*}}}
\newcommand{\leqLst}{\mathrel{\preceq_{L^*}}}
\newcommand{\lLst}{\mathrel{\prec_{L^*}}}

%\newcounter{lecture}
%\newcommand{\lecturefl}[1]{   % use with foiltex landscape
%% \addtocounter{lecture}{1}
% \refstepcounter{lecture}
% \setcounter{equation}{0}
% \setcounter{page}{1}
% \renewcommand{\theequation}{\arabic{equation}}
% \renewcommand{\thepage}{\arabic{lecture}--\arabic{page}}
% \raggedright
% \parindent 0pt
% \rightfooter{\thepage}
% \leftheader{}
% \rightheader{}
% \LogoOff
% \input header 
% \begin{center}
%% {\Large \bfseries Lecture \arabic{lecture} \\*[\bigskipamount] {#1}}
%{\Large \bfseries \arabic{lecture}.  {#1}}
% \end{center}
% \MyLogo{#1}
%}

%\newcommand{\lectureflstar}[1]{   % use with foiltex landscape
% \setcounter{equation}{0}
% \setcounter{page}{1}
% \renewcommand{\theequation}{\arabic{equation}}
% \renewcommand{\thepage}{\arabic{page}}
% \raggedright
% \parindent 0pt
% \rightfooter{\thepage}
% \leftheader{}
% \rightheader{}
% \LogoOff
% \input header 
% \begin{center}
% {\Large \bfseries #1}
% \end{center}
% \MyLogo{#1}
%}
%\newcounter{oursection}
%\newcommand{\frametitle}[1]{  % for use with foiltex landscape
% \addtocounter{oursection}{1}
%% \setcounter{equation}{0}
% \foilhead[-1.0cm]{#1}
% \LogoOn
%}

\newenvironment{algdesc}%
   {\begin{list}{}{%
   \setlength{\rightmargin}{0\linewidth}%
   \setlength{\leftmargin}{.05\linewidth}}%
   \sffamily\small
   \item[]{\setlength{\parskip}{0ex}\hrulefill\par%
   \nopagebreak{}}}%
   {{\setlength{\parskip}{-1ex}\nopagebreak\par\hrulefill} \end{list}}

\newenvironment{colm}{\left[\begin{array}{c}}{\end{array}\right]}
\newenvironment{colv}{\left(\begin{array}{c}}{\end{array}\right)}


\newcommand{\oh}{\frac12}
\newcommand{\st}{\text{subject to}}
\newcommand{\gfb}{\nabla f(\bar x)}
\newcommand{\hfb}{H(\bar x)}

\newtheorem{theorem}{Theorem}[section]
\newtheorem{remark}[theorem]{Remark}%[section]
\newtheorem{definition}[theorem]{Definition}%[section]
\newtheorem{proposition}[theorem]{Proposition}%[section]
\newtheorem{lemma}[theorem]{Lemma}%[section]
\newtheorem{corollary}[theorem]{Corollary}%[section]
\newtheorem{assumption}{Assumption}
\newtheorem{claim}{Claim}
\newtheorem{exam}{Example}
\newenvironment{solution}
  {\renewcommand\qedsymbol{$\square$}\begin{proof}[\textbf{Solution}]}
  {\end{proof}}
\renewcommand{\proofname}{\textbf{Proof}}

% Colors
\newcommand{\red}[1]{\textcolor{red}{#1}}
\newcommand{\blue}[1]{\textcolor{blue}{#1}}
\newcommand{\green}[1]{\textcolor{green}{#1}}

% Handy math notations
\newcommand{\grad}{\nabla}
\newcommand{\hess}{\nabla^2}
\newcommand{\tr}{\text{tr}}
\newcommand{\pd}[2][]{ \frac{\partial #1}{\partial #2}} % Partial derivatives
\renewcommand{\d}{{\rm d}}
\newcommand{\ddt}{\frac{\d}{\d t}}
\newcommand{\half}{\frac{1}{2}} % 1/2
\newcommand{\inv}{^{-1}}        % inverse
\newcommand{\T}{^\top}          % transpose

\begin{document}
\title{IOE 611: Homework 5}
\author{Yulun Zhuang}
\maketitle
%**********************************
\section*{Problem 1}
\textit{A convex problem in which strong duality fails}. Consider the optimization problem
\begin{align*}
  \text{minimize} \quad & e^{-x}\\
  \text{subject to} \quad & x^2/y \leq 0
\end{align*}
with variables $x$ and $y$, and domain $\mathcal{D} = \{(x, y)\mid y>0\}$.

(a) Verify that this is a convex optimization problem. Find the optimal value.

(b) Give the Lagrange dual problem, and find the optimal solution $\lambda^*$ and optimal value $d^*$ of the dual problem. What is the optimal duality gap?

(c) Does Slater's condition hold for this problem?

(d) What is the optimal value $p^*(u)$ of the perturbed problem
\begin{align*}
  \text{minimize} \quad & e^{-x}\\
  \text{subject to} \quad & x^2/y \leq u
\end{align*}
as a function of $u$? Verify that the global sensitivity inequality
\[
  p^*(u) \geq p^*(0) - \lambda^* u
\]
does not hold.



\clearpage
\section*{Problem 2}
\textit{Geometric interpretation of duality}. For each of the following optimization problems, draw a sketch of the sets
\begin{align*}
  \mathcal{G} &= \{(u, t)\mid \exists x\in \mathcal{D}, f_0(x) = t, f_1(x) = u\},\\
  \mathcal{A} &= \{(u, t)\mid \exists x\in \mathcal{D}, f_0(x) \leq t, f_1(x) \leq u\},
\end{align*}
give the dual problem, and solve the primal and dual problems. Is the problem convex? Is Slater's condition satisfied? Does strong duality hold?

The domain of the problem is $\real$ unless otherwise stated.

(a) Minimize $x$ subject to $x^2 \leq 1$.

(b) Minimize $x$ subject to $x^2 \leq 0$.

(c) Minimize $x$ subject to $|x| \leq 0$.

(d) Minimize $x$ subject to $f_1(x) \leq 0$ where
\begin{align*}
  f_1(x) = 
  \begin{cases}
    -x + 2 & x \geq 1\\
    x & -1 \leq x \leq 1\\
    -x-2 & x\leq -1.
  \end{cases}
\end{align*}


(e) Minimize $x^3$ subject to $-x + 1 \leq 0$.

(f) Minimize $x^3$ subject to $-x + 1 \leq 0$ with domain $\mathcal{D} = \real_+$.


\clearpage
\section*{Problem 3}
\textit{Equality constrained least-squares}. Consider the equality constrained least-squares problem
\begin{align*}
  \text{minimize} \quad & \|Ax - b\|^2_2\\
  \text{subject to} \quad & Gx = h
\end{align*}
where $A\in\real^{m\times n}$ with $\Rank A = n$, and $G \in\real^{p\times n}$ with $\Rank G = p$.

Give the KKT conditions, and derive expressions for the primal solution $x^*$ and the dual solution $\nu^*$.



\clearpage
\section*{Problem 4}
Derive the KKT conditions for the problem
\begin{align*}
  \text{minimize} \quad & \Tr X - \log\det X\\
  \text{subject to} \quad & Xs = y,
\end{align*}
with variable $X\in\symm^n$ and domain $\symm^n_{++}$. $y\in\real^n$ and $s\in\real^n$ are given, with $s\T y = 1$.

Verify that the optimal solution is given by
\[
  X^* = I + yy\T - \frac{1}{s\T s} s s\T.
\]


\clearpage
\section*{Problem 5}
\textit{Dual of SOCP}. Show that the dual of the SOCP
\begin{align*}
  \text{minimize} \quad & f\T x\\
  \text{subject to} \quad & \|A_i x + b_i\|_2 \leq c_i\T x + d_i, \quad i = 1, \dots, m,
\end{align*}
with variables $x\in\real^n$, can be expressed as 
\begin{align*}
  \text{minimize} \quad & \sum_{i=1}^{m} (b_i\T u_i - d_i v_i)\\
  \text{subject to} \quad & \sum_{i=1}^{m} (A_i\T u_i - c_i v_i) + f = 0\\
  & \|u_i\|_2 \leq v_i, \quad i = 1, \dots, m,
\end{align*}
with variables $u_i\in\real^{n_i}, v_i\in\real, i = 1, \dots, m$. The problem data are $f\in\real^n, A_i\in\real^{n_i\times n}$.

Derive the dual in the following two ways.

(a) Introduce new variables $y_i\in\real^{n_i}$ and $t_i \in \real$ and equalities $y_i = A_i x + b_i, t_i = c_i\T x + d_i$, and derive the Lagrange dual.

(b) Start from the conic formulation of the SOCP and use the conic dual. Use the fact that the second-order cone is self-dual.



\clearpage
\section*{Problem 6}
\textit{Gradient and Newton methods}. Consider the unconstrained problem
\[
\text{minimize} \quad f(x) = -\sum_{i=1}^{m}\log(1 - a_i\T x) - \sum_{i=1}^{n} \log(1 - x^2_i),
\]
with variable $x\in\real^n$, and $\dom f=\{x \mid a_i\T x < 1, i = 1, \dots, m, |x_i| < 1, i = 1, \dots, n\}$.
This is the problem of computing the analytic center of the set of linear inequalities
\[
a_i\T x \leq 1, \quad i=1, \dots, m, \quad |x_i| \leq 1, \quad i = 1, \dots, n.
\]

Note that we can choose $x^{(0)} = 0$ as our initial point. You can generate instances of this problem by choosing $a_i$ from some distribution on $\real^n$.

(a) Use the gradient method to solve the problem, using reasonable choices for the backtracking parameters, and a stopping criterion of the form $\|\grad f(x) \|_2 \leq \eta$. Plot the objective function and step length versus iteration number. (Once you have determined $p^*$ to high accuracy, you can also plot $f - p^*$ versus iteration.) Experiment with the backtracking parameters $\alpha$ and $\beta$ to see their effect on the total number of iterations required. Carry these experiments out for several instances of the problem, of different sizes.

(b) Repeat using Newton's method, with stopping criterion based on the Newton decrement $\lambda^2$. Look for quadratic convergence. You do not have to use an efficient method to compute the Newton step; you can use a general purpose dense solver, although it is better to use one that is based on a Cholesky factorization.



\end{document}
