\documentclass{article}

\usepackage{amsmath, amsfonts, amsthm, amssymb} 
\usepackage{listings}
\usepackage{graphicx}
\usepackage{float}
\usepackage{subfigure}
\usepackage{geometry}
\usepackage{hyperref}
\usepackage[parfill]{parskip} % no newline indent
\usepackage{enumitem} % enumerate / ordered list
\usepackage{booktabs} % three-line table
\usepackage{array}   % for \newcolumntype macro
\newcolumntype{C}{>{$}c<{$}} % math-mode version of "l" column type

\theoremstyle{definition} % difinition
\newtheorem{definition}{Definition}[section]
\newtheorem{theorem}{Theorem}[section]
\newtheorem{remark}{Remark}[section]

\newcommand{\dd}{\mathrm{d}}
\newcommand{\RR}{\mathbb{R}}
\newcommand{\NN}{\mathbb{N}}
\newcommand{\ZZ}{\mathbb{Z}}

\geometry{
	paper=a4paper, 
	top=2.5cm,
	bottom=2.5cm, 
	left=2.5cm, 
	right=3cm,
	headsep=0.75cm, 
}
\title{ROB 501 HW4}
\author{Yulun Zhuang \\ \href{mailto:yulunz@umich.edu}{yulunz@umich.edu}}
\date{\today}

\begin{document}

\maketitle

\section{}

\begin{align*}
    \alpha_1 
    \begin{bmatrix}
        1 \\ 2 \\ -1 \\ 3
    \end{bmatrix}
    +
    \alpha_2 
    \begin{bmatrix}
        1 \\ 0 \\ 0 \\ 2
    \end{bmatrix}
    +
    \alpha_3 
    \begin{bmatrix}
        2 \\ 8 \\ -4 \\ 8
    \end{bmatrix}
    +
    \alpha_4 
    \begin{bmatrix}
        1 \\ 1 \\ 1 \\ 1
    \end{bmatrix}
    +
    \alpha_5
    \begin{bmatrix}
        3 \\ 3 \\ 0 \\ 6
    \end{bmatrix}
    &= 0
    \\
    \underbrace{
    \begin{bmatrix}
        1 & 1 & 2 & 1 & 3 \\
        2 & 0 & 8 & 1 & 3 \\
        -1 & 0 & -4 & 1 & 0 \\
        3 & 2 & 8 & 1 & 6 \\
    \end{bmatrix}
    }_A
    \begin{bmatrix}
        \alpha_1 \\ \alpha_2 \\ \alpha_3 \\ \alpha_4 \\ \alpha_5
    \end{bmatrix}
    &= 0
\end{align*}
Since $rank(A)$ is 3, and through observations
\begin{align*}
    \begin{bmatrix}
        2 \\ 8 \\ -4 \\ 8
    \end{bmatrix}
    &=
    -6
    \begin{bmatrix}
        1 \\ 0 \\ 0 \\ 2
    \end{bmatrix}
    -4
    \begin{bmatrix}
        1 \\ 1 \\ 1 \\ 1
    \end{bmatrix}
    4
    \begin{bmatrix}
        3 \\ 3 \\ 0 \\ 6
    \end{bmatrix}
    \\
    \begin{bmatrix}
        1 \\ 2 \\ -1 \\ 3
    \end{bmatrix}
    &=
    -
    \begin{bmatrix}
        1 \\ 0 \\ 0 \\ 2
    \end{bmatrix}
    -
    \begin{bmatrix}
        1 \\ 1 \\ 1 \\ 1
    \end{bmatrix}
    +
    \begin{bmatrix}
        3 \\ 3 \\ 0 \\ 6
    \end{bmatrix}
\end{align*}

The dimension of the space spanned by columns of $A$ is 3.

\section{}

\begin{align*}
    x &= 
    \begin{bmatrix}
        8 \\ 7 \\ 4
    \end{bmatrix}
    = 8e_1 + 7e_2 + 4e_3
    \\
    [x]_S &= 
    \begin{bmatrix}
        8 \\ 7 \\ 4        
    \end{bmatrix}
\end{align*}

\begin{align*}
    &x = \alpha_1 u_1 + \alpha_2 u_2 + \alpha_3 u_3 = 
    \begin{bmatrix}
        8 \\ 7 \\ 4
    \end{bmatrix}
    \\
    &\left\{
        \begin{array}{lll}
            \alpha_1 + \alpha_2 + \alpha_3 = 8\\
            \alpha_1 + 2\alpha_2 + 2\alpha_3 = 7\\
            \alpha_1 + 2\alpha_2 + 3\alpha_3 = 4
        \end{array}
    \right. 
    \\
    \Rightarrow
    &\left\{
        \begin{array}{lll}
            \alpha_1 = 9\\
            \alpha_2 = 2\\
            \alpha_3 = -3
        \end{array}
    \right. 
    \\
    &[x]_U = 
    \begin{bmatrix}
        9 \\ 2 \\ -3
    \end{bmatrix}
\end{align*}

\section{}
\begin{theorem}
    There exists an invertible matrix $P$, with colefficients in $\mathcal{F}$, such that $\forall x \in(\mathcal{X}, \mathcal{F}),[x]_{\bar{u}}=P[x]_u$, where, $P=$ $\left[\begin{array}{llll}P_1 & P_2 & \cdots & P_n\end{array}\right]$ and its $i^{t h}$ column is given by $P_i:=\left[u^i\right]_{\bar{u}} \in \mathcal{F}^n$, and $\left[u^i\right]_{\bar{u}}$ is the representation of $u^i$ with respect to $\bar{u}$. Similarly, there exists an invertible matrix $\bar{P}=\left[\begin{array}{llll}\bar{P}_1 & \bar{P}_2 & \cdots & \bar{P}_n\end{array}\right]$ with $\bar{P}_i=\left[\bar{u}^i\right]_u$, the representation of $\bar{u}^i$ with respect to u, and $P \cdot \bar{P}=\bar{P} \cdot P=I$
\end{theorem}


From standard basis to the new basis, we have
\begin{align*}
    &s_1 = 2 u_1 - u_2\\
    &s_2 = - u_1 + 2u_2 - u_3\\
    &s_3 = -u_2 + u_3
\end{align*}

Hence,
\begin{align*}
    P = 
    \begin{bmatrix}
        2 & -1 & 0 \\
        -1 & 2 & -1 \\
        0 & -1 & 1
    \end{bmatrix}
\end{align*}

\section{}
Find the change of basis matrix $P$ from the world frame $(X_W, Y_W)$ to the robot's frame $(X_R, Y_R)$, such that $[x]_R = P[x]_W$.

\begin{align*}
    &P_i = [w_i]_R\\
    &w_1 = [1, 0]^T,\ w_2 = [0, 1]^T\\
    &[w_1]_R = [cos\theta, -sin\theta]^T,\ [w_2]_R = [sin\theta, cos\theta]^T\\
    \Rightarrow
    &P = 
    \begin{bmatrix}
        cos\theta & sin\theta\\
        -sin\theta & cos\theta
    \end{bmatrix}
\end{align*}

\section{}

\subsection{}
To show that $M$ is a basis of $\RR^{2, 2}$, we have to show
\begin{enumerate}[label=(\alph*)]
    \item $M$ is linear independent
    \item $span\{M\}=\RR^{2,2}$
\end{enumerate}

\begin{proof}
    For part (a)
    \begin{align*}
        &\alpha_1 M_1 + \alpha_2 M_2 + \alpha_3 M_3 + \alpha_4 M_4 = 0
        \\
        \Rightarrow\ 
        &\alpha_1 = \alpha_2 = \alpha_3 = \alpha_4 = 0
    \end{align*}
    For part (b), given an arbitrary element $U$ in $\RR^{2,2}$
    \begin{align*}
        U = 
        \begin{bmatrix}
            u_{11} & u_{12}\\
            u_{21} & u_{22}
        \end{bmatrix}
    \end{align*}
    It can be written as a linear combination by elements in $M$.
    \begin{align*}
        U =& \alpha_1 M_1 + \alpha_2 M_2 + \alpha_3 M_3 + \alpha_4 M_4 \\
        \begin{bmatrix}
            u_{11} & u_{12}\\
            u_{21} & u_{22}
        \end{bmatrix}
        =& 
        \alpha_1
        \begin{bmatrix}
            0 & 1\\
            1 & 0
        \end{bmatrix}
        +
        \alpha_2
        \begin{bmatrix}
            0 & -1\\
            1 & 0
        \end{bmatrix}
        +
        \alpha_3
        \begin{bmatrix}
            1 & 0\\
            0 & 1
        \end{bmatrix}
        +
        \alpha_4
        \begin{bmatrix}
            1 & 0\\
            0 & -1
        \end{bmatrix}
        \\
        =&
        \begin{bmatrix}
            \alpha_3 + \alpha_4 & \alpha_1 - \alpha_2\\
            \alpha_1 + \alpha_2 & \alpha_3 - \alpha_4
        \end{bmatrix}
        \\
        \Rightarrow\ 
        & \left\{
        \begin{array}{lll}
            \alpha_1 = (u_{12} + u_{21})/2\\
            \alpha_2 = (u_{21} - u_{12})/2\\
            \alpha_3 = (u_{11} + u_{22})/2\\
            \alpha_4 = (u_{11} - u_{22})/2\\
        \end{array}
        \right. 
    \end{align*}
    Hence, $span\{M\}=\RR^{2,2}$.
\end{proof}

\subsection{}
Given
$$
A = 
\begin{bmatrix}
    1 & 2 \\ 3 & 4
\end{bmatrix}
$$
We have

\begin{align*}
    [A]_S = 
    \begin{bmatrix}
        1 \\ 2 \\ 3 \\ 4
    \end{bmatrix}
\end{align*}

To find the change of basis matrix $P$ ($[A]_M = P[A]_S$) from standard basis $S$ to new basis $M$, we compute $\bar P$ where $P\bar{P}^{-1} = I$ and $\bar{P_i} = [m_i]_S$.
\begin{align*}
    [M_1]_S = 
    \begin{bmatrix}
        0 \\ 1 \\ 1 \\ 0
    \end{bmatrix},
    [M_2]_S &= 
    \begin{bmatrix}
        0 \\ -1 \\ 1 \\ 0
    \end{bmatrix},
    [M_3]_S = 
    \begin{bmatrix}
        1 \\ 0 \\ 0 \\ 1
    \end{bmatrix},
    [M_4]_S = 
    \begin{bmatrix}
        1 \\ 0 \\ 0 \\ -1
    \end{bmatrix}
    \\
    \Rightarrow
    \bar P &= 
    \begin{bmatrix}
        0 & 0 & 1 & 1\\
        1 & -1 & 0 & 0 \\
        1 & 1 & 0 & 0\\
        0 & 0 & 1 & -1
    \end{bmatrix}
    \\
    \Rightarrow
    P &= \bar{P}^{-1} =
    \begin{bmatrix}
        0 & 0.5 & 0.5 & 0\\
        0 & -0.5 & 0.5 & 0 \\
        0.5 & 0 & 0 & 0.5\\
        0.5 & 0 & 0 & -0.5
    \end{bmatrix}
    \\
    \Rightarrow
    [A]_M &= P[A]_S = 
    \begin{bmatrix}
        2.5 \\ 0.5 \\ 2.5 \\ -1.5
    \end{bmatrix}
\end{align*}

\section{}

\subsection{}
Find the representation of $r(x)$ w.r.t. basis $S$.

\begin{align*}
    r(x) &= 2 + 3x- x^2 = 2p_0 + 3p_1 -p_2\\
    [r(x)]_S &= 
    \begin{bmatrix}
        2 \\ 3 \\ -1
    \end{bmatrix}
\end{align*}

\subsection{}
Find the representation of $r(x)$ w.r.t. basis $Q$.

\begin{align*}
    [r(x)]_Q &= P[r(x)]_S\\
    P_i &= [S_i]_Q\\
    P_1 = 
    \begin{bmatrix}
        1 \\ 0 \\ 0
    \end{bmatrix},
    P_2 &= 
    \begin{bmatrix}
        1 \\ -1 \\ 0
    \end{bmatrix},
    P_3 = 
    \begin{bmatrix}
        -1 \\ 1 \\ 1
    \end{bmatrix}
    \\
    \Rightarrow
    P &= 
    \begin{bmatrix}
        1 & 1 & -1\\
        0 & -1 & 1\\
        0 & 0 & 1
    \end{bmatrix}
\end{align*}

Hence, 
$$
[r(x)]_Q = 
\begin{bmatrix}
    1 & 1 & -1\\
    0 & -1 & 1\\
    0 & 0 & 1
\end{bmatrix}
\begin{bmatrix}
    2 \\ 3 \\ -1
\end{bmatrix}
=
\begin{bmatrix}
    6 \\ -4 \\ -1
\end{bmatrix}
$$

\section{}
Let $\mathcal{F}=\mathbb{R}$ and let $\mathcal{X}$ be the set of $2 \times 2$ matrices with real coefficients. Define $L: \mathcal{X} \rightarrow \mathcal{X}$ by

$$
L(M)=2\left(M+M^{\top}\right)
$$

\subsection{}
\begin{proof}
    \begin{align*}
        L(\alpha A+ \beta B) &= 2(\alpha A+ \beta B+\alpha A^T + \beta B^T)\\
        &= \alpha 2(A+A^T) + \beta 2(B+B^T)\\
        &= \alpha L(A) + \beta L(B)
    \end{align*}
\end{proof}

\subsection{}

\begin{definition}
    Let $(\mathcal{X}, \mathcal{F})$ and $(\mathcal{Y}, \mathcal{F})$ be finite dimensional vector spaces, and $\mathcal{L}: \mathcal{X} \rightarrow \mathcal{Y}$ be a linear operator. A matrix representation of $\mathcal{L}$ with respect to a basis $u:=\left\{u^1, \ldots, u^m\right\}$ for $\mathcal{X}$ and $v:=\left\{v^1, \ldots, v^n\right\}$ for $\mathcal{Y}$ is an $n \times m$ matrix $A$, with coefficients in $\mathcal{F}$, such that $\forall x \in \mathcal{X},[\mathcal{L}(x)]_v=A[x]_u$.
\end{definition}
\begin{theorem}
    Let $(\mathcal{X}, \mathcal{F})$ and $(\mathcal{Y}, \mathcal{F})$ be finite dimensional vector spaces, $\mathcal{L}: \mathcal{X} \rightarrow \mathcal{Y}$ a linear operator, $u:=\left\{u^1, \ldots, u^m\right\}$ a basis for $\mathcal{X}$ and $v:=\left\{v^1, \ldots, v^n\right\}$ a basis for $\mathcal{Y}$, then $\mathcal{L}$ has a matrix representation $A=\left[\begin{array}{lll}A_1 & \cdots & A_m\end{array}\right]$, where the $i^{\text {th }}$ column of $A$ is given by
    $$
    A_i:=\left[\mathcal{L}\left(u^i\right)\right]_v, 1 \leq i \leq m
    $$
\end{theorem}

\begin{align*}
    A_i &= [L(E_i)]_E\\
    L(E_1) &= 
    \begin{bmatrix}
        4 & 0\\
        0 & 0
    \end{bmatrix}
    = 4 E_1\\
    L(E_2) &= 
    \begin{bmatrix}
        0 & 2\\
        2 & 0
    \end{bmatrix}
    = 2E_2 + 2E_3\\
    L(E_3) &= 
    \begin{bmatrix}
        0 & 2\\
        2 & 0
    \end{bmatrix}
    = 2E_2 + 2E_3\\
    L(E_4) &= 
    \begin{bmatrix}
        0 & 0\\
        0 & 4
    \end{bmatrix}
    = 4 E_4
\end{align*}

\begin{align*}
    \Rightarrow
    A = 
    \begin{bmatrix}
        4 & 0 & 0 & 0\\
        0 & 2 & 2 & 0\\
        0 & 2 & 2 & 0\\
        0 & 0 & 0 & 4
    \end{bmatrix}
\end{align*}


\section{}
Let $A$ be an $n \times n$ matrix with possibly complex coefficients. Let $L: \mathbb{C}^n \rightarrow \mathbb{C}^n$ by $L(x)=A x$. Note that the field is $\mathcal{F}=\mathbb{C}$.

\subsection{}
Compute the matrix representation of $L$ when the "natural" (also called canonical) basis is used in $\mathbb{C}^n$. Call your representation $\hat{A}$ and find its relation to the original matrix $A$.

The matrix representation $\hat A$ of $L$ satisfies $[L(x)]_S = \hat{A}[x]_S$.

Define the natural basis in $\mathbb{C}^n$ to be
\begin{align*}
    S = \left\{
        e_1=
        \begin{bmatrix}
            1 \\ 0 \\ \vdots \\ 0
        \end{bmatrix},
        e_2=
        \begin{bmatrix}
            0 \\ 1 \\ \vdots \\ 0
        \end{bmatrix},
        \dots,
        e_n=
        \begin{bmatrix}
            0 \\ 0 \\ \vdots \\ 1
        \end{bmatrix}
        \right\}
\end{align*}

\begin{align*}
    \hat A &= [\hat A_1 \mid \hat A_2 \mid \dots \mid \hat A_n]\\
    \hat A_i &= [L(e_i)]_S\\
    &= [Ae_i]_S \Leftarrow L(x) = Ax\\
    &= [A_i]_S \Leftarrow e_i = [0, \dots, 1, \dots, 0]^T\\
    &= A_i\\
    \Rightarrow 
    \hat A &= A
\end{align*}

\subsection{}
Suppose that the e-values of $A$ are distinct. Compute the matrix representation $L$ with respect to a basis constructed from the e-vectors of $A$. Call your representation $\hat{A}$.

Recall $L(x)=Ax$, $Ax = \lambda x$ and the matrix representation $\hat A$ of $L$ satisfies $[L(x)]_S = \hat{A}[x]_S$.


Define a set of basis constructed from the e-vectors of $A$ to be

\begin{align*}
    V = \left\{v_1, v_2, \dots, v_n\right\}
\end{align*}

\begin{align*}
    \hat A &= [\hat A_1 \mid \hat A_2 \mid \dots \mid \hat A_n]\\
    \hat A_i &= [L(v_i)]_V\\
    &= [Av_i]_V \Leftarrow L(x) = Ax\\
    &= [\lambda_i v_i]_V \Leftarrow Ax = \lambda x\\
    &= \lambda_i[v_i]_V\\
    &= 
    \begin{bmatrix}
        0 \\ \vdots \\ \lambda_i \\ \vdots \\ 0
    \end{bmatrix}
    \\
    \hat A &= 
    \begin{bmatrix}
        \lambda_1 & 0 & \dots & 0\\
        0 & \lambda_2 & \dots & 0 \\
        \vdots & \vdots & \ddots& \vdots\\
        0 & 0 & \dots & \lambda_n\\
    \end{bmatrix}
\end{align*}


\end{document}