\documentclass{article}

\usepackage{amsmath, amsfonts, amsthm, amssymb} 
\usepackage{listings}
\usepackage{graphicx}
\usepackage{float}
\usepackage{subfigure}
\usepackage{geometry}
\usepackage{hyperref}
\usepackage[parfill]{parskip} % no newline indent
\usepackage{enumitem} % enumerate / ordered list
\usepackage{booktabs} % three-line table
\usepackage{array}   % for \newcolumntype macro
\newcolumntype{C}{>{$}c<{$}} % math-mode version of "l" column type

\theoremstyle{definition} % difinition
\newtheorem{definition}{Definition}[section]
\newtheorem{theorem}{Theorem}[section]
\newtheorem{remark}{Remark}[section]

\newcommand{\dd}{\mathrm{d}}
\newcommand{\RR}{\mathbb{R}}
\newcommand{\NN}{\mathbb{N}}
\newcommand{\ZZ}{\mathbb{Z}}

\geometry{
	paper=a4paper, 
	top=2.5cm,
	bottom=2.5cm, 
	left=2.5cm, 
	right=3cm,
	headsep=0.75cm, 
}
\title{ROB 501 HW3}
\author{Yulun Zhuang \\ \href{mailto:yulunz@umich.edu}{yulunz@umich.edu}}
\date{\today}

\begin{document}

\maketitle

\section{}

\begin{enumerate}[label=(\alph*)]
    \item Not a subspace. Reason: Not closed under multiplication by a constant, such as -1.
    \item A subspace. Reason: Closed under vector addition and scalar multiplication.
    \item Not a subspace. Reason: Not closed under addition, such as $[0, 1]^T + [1, 0]^T = [1, 1]^T$.
    \item A subspace. Reason: Closed under vector addition and scalar multiplication.
    \item Not a subspace. Reason: Zero vector is not included.
    \item Not a subspace. Reason: Zero vector is not included.
\end{enumerate}

\section{}

\begin{definition}
    Let $\mathcal S$ be a subset of a vector space($X$, $\mathcal{F})$. The span of $\mathcal S$ is the set of all linear combinations of elements of $\mathcal S$. That is
    $$span\{\mathcal{S}\}:=\left\{x \in X \mid \exists n \geq 1, \alpha_1, \ldots, \alpha_n \in \mathcal{F}, v_1, \ldots, v_n \in \mathcal{S}, \text { s.t. } x=\alpha_1 v_1+\cdots+\alpha_n v_n\right\}$$
\end{definition}

\begin{theorem}
    Given a finite set $\mathcal S$ in a vector space $X$, $span\{\mathcal S\}$ is a subspace of $X$.
\end{theorem}

\begin{proof}
    Let $u_1, \ldots, u_n \in \mathcal{S}_1$, $v_1, \ldots, v_n \in \mathcal{S}_2$, then $u_1, \ldots, u_n, v_1, \ldots, v_n \in \mathcal{S}_1 \cup \mathcal{S}_2 $.
    $$span\{\mathcal{S}_1\}:=\left\{u \in X \mid  u=\alpha_1 u_1+\cdots+\alpha_n u_n\right\}$$
    $$span\{\mathcal{S}_2\}:=\left\{v \in X \mid  v=\beta_1 v_1+\cdots+\beta_n v_n\right\}$$
    Then $span\{\mathcal{S}_1\}$ and $span\{\mathcal{S}_2\}$ are subspaces in $X$. The addition of the subspaces is given by
    $$span\{\mathcal{S}_1\}+span\{\mathcal{S}_2\} = \{w \in X\mid w = \alpha_1 u_1+\cdots+\alpha_n u_n + \beta_{1} v_1+\cdots+\beta_{n}v_n\}$$
    Since $u_1, \ldots, u_n, v_1, \ldots, v_n \in \mathcal{S}_1 \cup \mathcal{S}_2 $, we have
    $span\left\{S_1\right\}+span\left\{S_2\right\} \subset span\left\{S_1 \cup S_2\right\}$ . \\
    $$span\{\mathcal{S}_1 \cup  \mathcal{S}_2\}:=\left\{z \in X \mid  z=\gamma_1 z_1+\cdots+\gamma_n z_n\right\}$$
    Since $z$ can be represented by a linear combination of $u$ and $v$, we have $span\left\{S_1 \cup S_2\right\} \subset span\left\{S_1\right\}+span\left\{S_2\right\}$.
    
    Hence $span\left\{S_1 \cup S_2\right\} = span\left\{S_1\right\}+span\left\{S_2\right\}$.
\end{proof}

\section{}
\subsection{}
Linear dependent
\begin{align*}
    \begin{bmatrix}
        1\\5\\9
    \end{bmatrix}
    =
    3
    \begin{bmatrix}
        1\\2\\3    
    \end{bmatrix}-
    \begin{bmatrix}
        2\\1\\0
    \end{bmatrix}
\end{align*}

\subsection{}
Linear dependent
\begin{align*}
    \begin{bmatrix}
        0\\5\\4
    \end{bmatrix}
    =
    4
    \begin{bmatrix}
        1\\2\\3    
    \end{bmatrix}
    - \frac{1}{2}
    \begin{bmatrix}
        0\\0\\6
    \end{bmatrix}
    - 4
    \begin{bmatrix}
        1\\1\\1   
    \end{bmatrix}
\end{align*}

\subsection{}
Linear independent, since the third component of the first vector can not be expressed as a linear combination of the rest.


\section{}
\begin{align*}
    \alpha_1 
    \begin{bmatrix}
        1 & 2 \\ 2 & 1
    \end{bmatrix}
    +
    \alpha_2
    \begin{bmatrix}
        2 & 1 \\ 1 & 1
    \end{bmatrix}
    +
    \alpha_3
    \begin{bmatrix}
        4 & -1 \\ -1 & 1
    \end{bmatrix}
    = \mathbf 0\\
    \begin{bmatrix}
        \alpha_1 + 2\alpha_2 + 4\alpha_3 & 2\alpha_1 + \alpha_2 - \alpha_3\\
        2\alpha_1 + \alpha_2 - \alpha_3 & \alpha_1 + \alpha_2 + \alpha_3
    \end{bmatrix}
    = \mathbf0\\
    \underbrace{
    \begin{bmatrix}
        1 & 2 & 4\\
        2 & 1 & -1\\
        1 & 1 & 1
    \end{bmatrix}}_A
    \underbrace{
    \begin{bmatrix}
        \alpha_1 \\ \alpha_2 \\ \alpha_3
    \end{bmatrix}}_x
    = \mathbf 0\\
    \Rightarrow \ det(A) = 0
\end{align*}
Choose $\alpha_1 = 2$, $\alpha_2 = -3$, $\alpha_1 = 1$ solved $Ax = \mathbf 0$. Hence the set is linearly dependent.

\section{}
Let ($X$, $\mathcal F$) be a vector space and $\mathcal S \subset {X}$. Prove that if $Y$ is a subspace of $X$ and $\mathcal S \subset Y$, then $span\{\mathcal{S}\}\subset Y$.

\begin{proof}
    $$span\{\mathcal{S}\}:=\left\{x \in X \mid \exists n \geq 1, \alpha_1, \ldots, \alpha_n \in \mathcal{F}, v_1, \ldots, v_n \in \mathcal{S}, \text { s.t. } x=\alpha_1 v_1+\cdots+\alpha_n v_n\right\}$$\\
    Since $S\subset Y$, then $v_1, \ldots, v_n \in Y$. Since $Y$ is closed under vector multiplication and scalar addition, $x\in Y$. Hence, $span\{\mathcal{S}\}\subset Y$.
\end{proof}

\section{}

Let ($X$, $\mathcal F$) be a vector space and $V$ and $W$ are subspaces of $X$. Prove the following two statements are equivalent.

\begin{enumerate}[label=(\alph*)]
    \item $V \cap W = \{0\}$
    \item $\forall x\in V+W, \exists$ unique $v\in V$ and $ w\in W$ s.t. $x = v+w$ 
\end{enumerate}

\begin{proof}
    ($a \Rightarrow b$) Since for every $x\in V+W$, there exist $v_1\in V$ and $ w_1\in W$ s.t. $x = v_1 + w_1$. Suppose there exist other vectors $v_2 \in V$ and $w_2 \in W$ such that $x = v_2 + w_2$. Then,
    $$\mathbf 0 = (v_1 - v_2) + (w_1 - w_2) \Leftrightarrow (v_1 - v_2) = - (w_1 - w_2)$$
    Therefore $v_1 - v_2 \in W$ and so $v_1 - v_2 \in V\cap W$. Since $V\cap W = \{0\}$, we then conclude that $v_1 = v_2$, which also says $w_1 = w_2$. Then $\forall x\in V+W, \exists$ unique $v\in V$ and $ w\in W$ s.t. $x = v+w$.\\
    ($b \Rightarrow a$) Suppose that $x_0 \in V\cap W$, then on the one hand, there exists $v_0\in V$ such that $x_0 = v_0 + \mathbf 0$; on the other hand, there is $w_0 \in W$ such that $x_0 = \mathbf 0 + w_0$. Therefore, $v_0 = \mathbf 0$ and $w_0 = \mathbf 0$, so $V \cap W = \{0\}$.
\end{proof}

\end{document}