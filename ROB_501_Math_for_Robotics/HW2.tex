\documentclass{article}

\usepackage{amsmath, amsfonts, amsthm, amssymb} 
\usepackage{listings}
\usepackage{graphicx}
\usepackage{float}
\usepackage{subfigure}
\usepackage{geometry}
\usepackage{hyperref}
\usepackage[parfill]{parskip} % no newline indent
\usepackage{enumitem} % enumerate / ordered list
\usepackage{booktabs} % three-line table
\usepackage{array}   % for \newcolumntype macro
\newcolumntype{C}{>{$}c<{$}} % math-mode version of "l" column type

\theoremstyle{definition} % difinition
\newtheorem{definition}{Definition}[section]

\newcommand{\dd}{\mathrm{d}}
\newcommand{\RR}{\mathbb{R}}
\newcommand{\NN}{\mathbb{N}}
\newcommand{\ZZ}{\mathbb{Z}}

\geometry{
	paper=a4paper, 
	top=2.5cm,
	bottom=2.5cm, 
	left=2.5cm, 
	right=3cm,
	headsep=0.75cm, 
}
\title{ROB 501 HW2}
\author{Yulun Zhuang \\ \href{mailto:yulunz@umich.edu}{yulunz@umich.edu}}
\date{\today}

\begin{document}

\maketitle

\section{}

\begin{table}[H]%[htb]
        \begin{center}
            % \caption{}\label{table:action_space}
            \begin{tabular}{CCCC}
                \toprule
                P & Q & P \land Q & P \lor Q \\
                \midrule
                T & T & T       & T      \\
                T & F & F       & T      \\
                F & T & F       & T      \\
                F & F & F       & F     \\
                \bottomrule
            \end{tabular}
        \end{center}
\end{table}


\subsection{}
$$
\neg (P\land Q) = \neg P \lor \neg Q
$$

\begin{table}[H]%[htb]
        \begin{center}
            % \caption{}\label{table:action_space}
            \begin{tabular}{CCC}
                \toprule
                \neg P & \neg Q & \neg P \lor \neg Q\\
                \midrule
                F & F & F       \\
                F & T & T       \\
                T & F & T       \\
                T & T & T       \\
                \bottomrule
            \end{tabular}
        \end{center}
\end{table}

\subsection{}
$$
\neg (P\lor Q) = \neg P \land \neg Q
$$

\begin{table}[H]%[htb]
        \begin{center}
            % \caption{}\label{table:action_space}
            \begin{tabular}{CCC}
                \toprule
                \neg P & \neg Q & \neg P \lor \neg Q\\
                \midrule
                F & F & F       \\
                F & T & F       \\
                T & F & F       \\
                T & T & T       \\
                \bottomrule
            \end{tabular}
        \end{center}
\end{table}

\section{}

\begin{enumerate}[label=(\alph*)]
    \item There exist an interger $n$, $2n+1$ is even.
    \item For every interger $n$, $2^n + 1$ is composite.
    \item $\forall v \in \RR^n, v\neq 0$ s.t. $Av\neq \lambda v$
    \item $\exists \eta > 0, \forall \delta> 0$ s.t. $|x|\le \delta \Rightarrow |f(x)|>\eta |x|$
\end{enumerate}

\section{}
Prove that $\sqrt{7}$ is irrational.

\begin{proof}
    
Assum $\sqrt{7}$ is rational.
Then there exist natural numbers $m$ and $n$ such that $m$ and $n$ have no common factor, $n\neq 0$ and 
\begin{equation}
    \sqrt{7} = \frac{m}{n}
    \label{eq:3-1}
\end{equation}

Square both sides of \eqref{eq:3-1}

\begin{equation}
    (7n^2 = m^2)\label{eq:3-2}
\end{equation}

This implies 7 divides $m^2$, so 7 also divides $m$.

Let $m = 7k, k\in\NN$,
\begin{align*}
    7n^2 =& (7k)^2\\
    n^2 =& 7k^2
\end{align*}
This implies 7 divides $n^2$, so 7 also divides $n$.

Therefore, $m$ and $n$ both have a common factor 7, which leads to a contradiction.

Hence, $\sqrt{7}$ is irrational.

\end{proof}

\section{}
Let $A$ be a square matrix. Prove that if $det(A)=0$, then $A$ is not invertible.


\begin{definition}
    A square matrix $A$ has an inverse $A^{-1}$ iff $AA^{-1} = A^{-1}A=I$.
\end{definition}

\begin{proof}

p: $det(A)=0$, q: $A$ is not invertible

Now we prove its contrapositive, $\neg q\implies\neg p$.

We have $A$ is invertible,

\begin{align*}
    det(AA^{-1}) = &det(A) det(A^{-1}) = det(I) = 1\\
    \Rightarrow \quad &det(A) \neq 0
\end{align*}

\end{proof}

\section{}
Prove that for all integers $n\ge 1$, $\sum^n_{k=1}\frac{1}{k(k+1)}=\frac{n}{n+1}$.

\begin{proof}
$P(n) = \sum^n_{k=1}\frac{1}{k(k+1)}=\frac{n}{n+1}$

When $n=1$, $P(1) = \frac{1}{1+1} = \frac{1}{2}$

When $n=m$, assum $P(m) = \sum^m_{k=1}\frac{1}{k(k+1)}=\frac{m}{m+1}$ is true.

When $n=m+1$,

\begin{align*}
    P(m+1) =& P(m) + \frac{1}{(m+1)(m+2)}\\
    =& \frac{m}{m+1} + \frac{1}{(m+1)(m+2)}\\
    =& \frac{(m+1)^2}{(m+1)(m+2)}\\
    =& \frac{m+1}{(m+1)+1}
\end{align*}

\end{proof}

\section{}

$P(n): \forall n\in \ZZ ,n\ge 12, \exists k_1, k_2 \in \ZZ, k_1, k_2 \ge 0, s.t.\ n=4k_1+5k_2$

\begin{proof}
If $n =12$, $k_1 = 3$ and $ k_2 = 0$.
If $n =13$, $k_1 = 2$ and $ k_2 = 1$.
If $n =14$, $k_1 = 1$ and $ k_2 = 2$.
If $n =15$, $k_1 = 0$ and $ k_2 = 3$.

Induction: Assume $P(n)$ is true for $12\le n\le k$, show that $P(k+1)$ is true.

Start with $k+1\ge 16$,

$$ (k+1)-4 \ge 12$$

By inductive hypothesis,

$$ (k+1)-4 = 4u+5v$$
where $u$ and $v$ are non-negtive integers.

Therefore,

$$ k+1 = 4(u+1) + 5v$$
\end{proof}

$P(n)$ is not true for $ n\ge 8$, since no combinations of '4' and '5' can add up to 11.

\subsection{}
$P(n): \forall n=2m, m\in \ZZ , n\ge 6, \exists k_1, k_2 \in \ZZ, k_1, k_2 \ge 0, s.t.\ n=3k_1+5k_2$

\begin{proof}
    If $n =6$, $k_1 = 2$ and $ k_2 = 0$.
    If $n =8$, $k_1 = 1$ and $ k_2 = 1$.
    If $n =10$, $k_1 = 0$ and $ k_2 = 2$.
    If $n =12$, $k_1 = 4$ and $ k_2 = 0$.
    
    Induction: Assume $P(n)$ is true for $6\le n\le k$, show that $P(k+1)$ is true.
    
    Start with $k+1\ge 12$,
    
    $$ (k+1)-6 \ge 6$$
    
    By inductive hypothesis,
    
    $$ (k+1)-6 = 3u+5v$$
    where $u$ and $v$ are non-negtive integers.
    
    Therefore,
    
    $$ k+1 = 3(u+2) + 5v$$
\end{proof}

\section{}
Given $M$ is an $n\times n$ real symmetric matrix. $f(x) = x^TMx$, subject to the constraint $x^Tx =1$.

Let $L(x, \lambda) = f(x) -\lambda (x^Tx-1) = x^TMx - \lambda x^Tx - \lambda$, and $\lambda > 0$.

$$\nabla L = \frac{dL}{dx} = 2Mx - 2\lambda x$$

Solve $\nabla L = 0$, we have $Mx = \lambda x$.

Therefore, $\lambda$ is the eigenvalue of $M$ and $x$ is the corresponding eigenvector.

$$
f(x) = x^TMx = x^T(\lambda x) = \lambda x^T x = \lambda
$$

Solve $det(M-\lambda I)=0$. Since the eigenvalues are real and finite in number, there exists a largest eigenvalue, denoted $\lambda_{max}$,
and a smallest eigenvalue, denoted  $\lambda_{min}$, i.e. $\boldsymbol \lambda = [\lambda_{max}, \dots, \lambda_{min}]^T$

\subsection{}

$max(f(x)) = \lambda_{max}$ and $x_{max}$ is its corresponding eigenvector given by $(M - \lambda_{max}I)x=0$.

\subsection{}


$min(f(x)) = \lambda_{min}$ and $x_{min}$ is its corresponding eigenvector given by $(M - \lambda_{min}I)x=0$.


\end{document}